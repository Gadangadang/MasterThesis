\section{Anomaly detection}
Anomaly detection is a versatile tool that finds application in a diverse range of scenarios, 
including fraud detection, anomalous sensor data analysis, and time series data. The primary 
objective of this tool is to identify data that deviates from a predetermined standard of 
normal behavior. The definition of this standard can vary from situation to situation, based 
on the context and the expected anomalous behavior. According to Chandola, Banerjee, and 
Kumar \cite{anom_detec}, anomalies can be classified into three categories: \textit{point anomalies}, 
\textit{contextual anomalies}, and \textit{collective anomalies}. \textit{Point anomalies} represent singular or few 
outliers from a larger group or context, and can occur in various situations. A notable example 
of a point anomaly is Michael Phelps, who is able to swim intensively for longer periods due 
to his body producing less lactic acid. \textit{Contextual anomalies}, on the other hand, are determined 
based on the context of the anomaly and data, rather than as a whole. For instance, in the 
case of continuous gas flow data, a sudden change in flow on a Saturday, despite being within 
the range of Friday's flow, could be categorized as a contextual anomaly. The third type, 
\textit{collective anomalies}, represents a group of anomalies that deviate from the expected behavior 
of the dataset. In particle physics experiments, collective anomalies are of particular 
interest as there are many sources of anomalous behavior, and only collective anomalies are 
worth investigating. Additionally, the noise generated by a large number of components in such 
experiments makes it essential to consider collective anomalies.
