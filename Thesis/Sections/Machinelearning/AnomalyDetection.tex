\section{Anomaly detection}
Anomaly detection is a tool with a wide range of uses, from time series data, fraud detection or anomalous sensor data. 
Its main purpose is to detect data which does not conform to some predetermined standard for normal behavior. 
The predetermined standard varies from situation to situation, both from the context it self and what is expected as an anomaly. 
Anomalies are typically classified in three categories \cite{anom_detec}:


\begin{enumerate}
    \item Point anomalies
    \item Contextual anomalies
    \item Collective anomalies
\end{enumerate}

Point anomalies are singular or few outliers from a larger context or group. These anomalies can occur in many situations, are 
indeed quite important to detect. One such example is Michael Phelps. Phelps is famous for being one of the best swimmers of all time. 
Along with extensive training, planning and dedication, he has another tool that has helped him, he does not produce much lactic acid. 
In fact, his body produces so little that he can swim continously and much more intensive than most other top swimmers. This ability 
is not common, in fact it is very rare amongst humans, and can be considered a point anomaly. It is important to understand that point anomalies
do not have to be singular occurances. Rather they are extremly rare events that deviate alot from the expected behavior. \par
Contextual anomalies are another kind of anomalies, and are defined based on the context of the anomaly and data, rather than as a whole. 
Suppose you have data on continous stream of gas in a pipe. The extraction of this gas is day dependent, to the point where the delivery 
on saturdays might oscillate between half and 3/4 of that of monday through friday, due to shorter work day. Should there one saturday 
suddenly flow the same amount as friday, an analyst think nothing of this, but due to this being a saturday, the context of this behavior 
dictates that this be categorized as a contextual anomaly. \par 
The last type of anomaly described by Chandola, Banerjee and Kumar \cite{anom_detec} are the collective anomalies. The collective anomalies 
are anomalies that as a group deviate from the expected behavior of the dataset. In particle physics these anomalies are the only type that 
are of interest. This is because there are so many sources for anomalous behavior in an experiment that only collective ones are worth investigating. 
The major problem for such experiments is noise, and noise can be created from a large number of components. This alone is reason enough to 
only consider collective anomalies. Another reason is that certain processes in particle physics look much alike, but have different cross section,
thus one process is much more likely to occur than another. This was one of the main issues with the discovery of the Higgs boson, as Higgs has a 
background  of 