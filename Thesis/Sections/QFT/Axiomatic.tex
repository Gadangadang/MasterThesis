\chapter{Introduction to Quantum Field Theory}\label{chap:Intro QFT}
In this chapter we aim to review some basic knowledge in quantum field theory. The idea is that before we tackle more advanced calculations, this chapter will serve as a smooth transition to later chapters. We assume that the reader is already acquainted with classical and quantum field theory to a basic level. Therefore, we will try to avoid too much detail and instead sketch the main ideas. 

We begin with a very brief discussion on why we need quantum field theory and the basic playing ground. From there we look at the canonical quantization procedure. This part will be very brief as we aim to use the path integral formulation of quantum field theory. We derive the path integral by looking at path integrals in quantum mechanics, where the transition to quantum field theory is made by replacing quantum mechanical operators with fields. After we have introduced path integrals, the path integral quantization procedure for free theories is explored. However, the main focus in this chapter is to explore Green's functions and how to relate them to scattering amplitudes and Feynman diagrams. Hence, we will explicitly show how one goes about formulating interacting theories using path integrals and perturbative expansions. For simplicity, we will mainly focus on scalar $\phi
^{4}$-theory as it serves as an easy example to introduce the main concepts. 

Lastly, we consider the consequences of quantum fluctuations and introduce the concepts of regularization and renormalization. The Callan-Symanzik equation will be derived and the notion of running coupling is discussed. We will try not to focus on all the details of calculations and rather state common results and discuss some important implications. For a more complete treatment on quantum field theory and its construction, see e.g. \cite{Peskin:257493,Schwartz:2013pla,Srednicki:2007qs}.




\section{Basic Construction and Lagrangians}
%The first obvious question is what a theory of quantum fields really is. Before trying to answer this question, let us step back a little and strip away the physical interpretation, and instead look at the abstract mathematical arena in which one performs calculations. For example, the theory of general relativity then becomes a theory on a pseudo-Riemannian manifold, non-relativistic quantum theory is the analysis of self-adjoint operators living in a Hilbert space. A quantum field theory is the attempt to unify quantum mechanics (QM) with special relativity (SR). Therefore, we have to consider spacetime dependent operators living in Hilbert space. For a free theory this is formally valid, but for interacting theories that is not the case. For example, perturbation theory is the main basis of performing calculations for an interacting field theory. What one generally does is to define a free Lagrangian $\mathcal{L}_{0}$, while the interactions is determined by a small \emph{perturbation} $\mathcal{L}_{int}$. The mathematical problem with this approach, is that there does not exist a single universal Hilbert space describing both free and interacting fields (\ar{ref to Wightman}). The consequence is that the central object we use in physics to describe scattering, namely the \emph{S-matrix}, is not formally a well defined object. However, this approach has been extremely successfull in desribing experimental results, so we will not concern ourselves with formal mathematical details.

\medskip
The first thing we want to do when constructing a theory is to define what space it lives in. In general, the space is a smooth Riemannian (or pseudo-Riemannian) manifold $M$ with general metric $g$. In particle physics, we have that $M$ is the four-dimensional Minkowski spacetime and $g$ is the Minkowski metric. There are a wide variety of interesting choices to study, but a general feature is that we always keep the metric fixed, as we would need quantum gravity to describe quantum fluctuations of the metric.

The next ingredient is to define what kind of fundamental objects that lives in this space. These objects are what we call \emph{fields}, and the simplest choice is a scalar field. A scalar field $\phi$ is just a function on $M$, which we can see as a map
\begin{align}
    \phi:\,M\rightarrow \mathbb{R}\, (\text{or}\,\mathbb{C})\,.
\end{align}

The main goal of quantum field theory is to construct a framework where special relativity (SR) and quantum mechanics (QM) are compatible. To that end, we have to look at a theory built on a Hilbert space $\mathcal{H}$ with states represented by unit rays and observables by self-adjoint operators, respecting the symmetry group of special relativity, the $\emph{Poincare}$ group. This is the group of Minkowski spacetime isometries, consisting of a semidirect product of the group of translations and the Lorentz group
\begin{align}
    \mathcal{P}=\mathbb{R}^{4}\rtimes SL(2,\mathbb{C})\,,
\end{align}
where $SL(2,\mathbb{C})$ is the universal cover of $SO^{+}(1,3)$, the group of proper orthocronous Lorentz transformations. It appears in quantum field theory since we need the representations to describe particles with spin. From \emph{Wigner's symmetry representation theorem} we view particles as irreducible representations of $\mathcal{P}$ on $\mathcal{H}$, and from \emph{Schur's lemma} we have that any irreducible representation can be classified using \emph{Casimir-invariants}. This leads to the remarkable result that particles can be classified only in terms of their mass and spin. 

To keep the next discussion as brief and simple as possible, we mainly consider real scalar fields, i.e. fields with spin zero. Let $\mathcal{C}$ denote the the space of field configurations on $M$, meaning every point $\phi\in \mathcal{C}$ corresponds to a configuration of the field. Typically $\mathcal{C}$ is an infinite-dimensional function space, and trying to understand the geometry and topology of this infinite space of fields, and then trying to do something useful with it is what makes quantum field theory so difficult, but at the same time so interesting. 


The first attempts of unifying SR with QM, physicists encountered problems like conservation of particles, negative energy states, violation of causality and Born's probability interpretation of QM. These problems could not be fixed in a consistent way using the known theories at the time. Thus, to reconcile these problems one needed a multi-particle theory, i.e a field theory.
To model this we must replace $\mathcal{H}$ with the direct sum of $n$-particle Hilbert spaces, also called the \emph{Fock space}
\begin{align}\label{eq:Fock space}
    \mathcal{H}\equiv\bigoplus_{n=0}^{\infty}S(\mathcal{H}^{\otimes n})\,,
\end{align}
symmetrized ($S$) as we are here describing bosonic fields. This construction is only formally valid for non-interacting fields, but let us not worry about such technicalities here\footnote{For more details on the axiomatic formulation of quantum field theory and \emph{Haag's theorem}, see e.g. \cite{inbook,klaczynski2016haags}.}. 

The next fundamental piece we need to determine is the \emph{action} for the theory. The action is a map from the configuration space to the real space
\begin{align}
    S[\phi]:\mathcal{C}\rightarrow \mathbb{R}\,,
\end{align}
meaning it is a real function on the space of fields, or in other words, given a field configuration, the action yields a real number. The action is what we call a \emph{functional}, just to remind ourselves that the configuration space is infinite-dimensional. From Hamilton's principle, we define the critical set
\begin{align}
    \text{Crit}_{\mathcal{C}}(S)=\{\phi\in\mathcal{C}\,|\,\delta S[\phi]=0\}\,,
\end{align}
which is a fancy way of saying that the fields in our theory obey the Euler-Lagrange equations. 

With a Minkowskian metric, we define the action as
\begin{align}\label{eq:basicaction}
    S[\phi]=\int d^{4}x\,\mathcal{L}(\phi,\partial_{\mu}\phi)\,,
\end{align}
where $\mathcal{L}$ is our Lagrangian\footnote{More correctly, it is the Lagrangian density, but this is standard terminology.} of the theory. So by applying Hamilton's principle $\delta S=0$, one finds the Euler-Lagrange equation
\begin{align}
    \partial_{\mu}\Big(\pdv{\mathcal{L}}{(\partial_{\mu}\phi)}\Big)-\pdv{\mathcal{L}}{\phi}=0\,,
\end{align}
and if the Lagrangian contains more than one field, there is one such equation for each.

With this brief introduction, we can now introduce a few important Lagrangians that we will use in this thesis.

\subsection*{Lagrangians}
Without reviewing classical field theory we will simply state the Lagrangians of interest and explore a couple of details that we will cover more on later. 

First of all, in any Lagrangian, we can group the different terms into three categories. These are kinetic terms, mass terms and interaction terms. The first two terms are quadratic in the same field, while the latter is either a combination of the same field or a combination of different fields. 

The kinetic term contains derivatives of the same field and describes the dynamics of that field, while the mass term describes the statics of this field. Upon quantizing a theory these terms will form what is called a \emph{propagator}, which we will cover the details on later. There are of course exceptions to the requirement that quadratic terms must involve the same field, but we will not cover this scenario here in this thesis.

Interaction terms are combinations of fields that are higher than quadratic in the fields. These terms are either a combination of the same field or a combination of different fields. Interactions are characterized by a constant of proportionality, which we call a \emph{coupling constant}. Eventually we want to have gauge invariant and renormalizable theories, which constrains the possible interaction terms. The restriction on renormalizable terms follow as the action is a dimensionless quantity\footnote{Lorentz invariance is always required.}. Hence, from \cref{eq:basicaction} the Lagrangian must have mass dimension four, and it can be shown that the coupling constant in renormalizable theories must be dimensionless. This has the implication that only field combinations with mass dimension four are \textquote{allowed}.

\subsubsection*{Real Scalar Field}
Let us begin with the simplest case of a Lagrangian describing a real scalar field\footnote{We also have complex scalar fields, but we keep this discussion as simple and brief as possible.}. A theory without interaction is what we call a \emph{free} theory, and for a real scalar field the Lagrangian can be written as
\begin{align}\label{eq:Lagrangian free scalar theory}
    \mathcal{L}_{S}=\frac{1}{2}(\partial_{\mu}\phi)(\partial^{\mu}\phi)-\frac{1}{2}m^{2}\phi^{2}\,,
\end{align}
where the minus sign in front of the \textquote{mass} term is crucial to keep the corresponding Hamiltonian bounded from below. We have to be careful with interpreting this as the physical mass of a field excitation. This is a crucial observation when we eventually encounter divergences in perturbation theory, but more on that later. The common factor of $1/2$ is just a convention that is chosen such that the propagator of the theory does not contain any factors in front.

By applying the Euler-Lagrange equation on this Lagrangian, we find the \emph{Klein-Gordon} equation
\begin{align}
    (\partial^{2}+m^{2})\phi(x)=0\,,
\end{align}  
where $\partial^{2}=\partial_{\mu}\partial^{\mu}$. We will study free theories and quantize them in the next sections, but they are not very interesting. That is not to say they are not useful to study, but ultimately they just describe a theory without interactions.

Let us look at a possible interaction term that we will explore in some detail in this chapter. From dimensional analysis we observe that $[\phi]=m$, implying that for a renormalizable scalar theory, we can only have interaction terms up to $\phi^{4}$. So if we add such a term to the free Lagrangian, we have the interacting theory called scalar $\phi^{4}$-theory
\begin{align}
    \mathcal{L}=\frac{1}{2}(\partial_{\mu}\phi)(\partial^{\mu}\phi)-\frac{1}{2}m^{2}\phi^{2}-\frac{\lambda}{4!}\phi^{4}\,,
\end{align}
where $\lambda$ is dimensionless and in \cref{sec:Renormalization} we will show how to formalise this as a renormalized theory.

By applying the Euler-Lagrange equation we find a inhomogenous differential equation
\begin{align}
    (\partial^{2}+m^{2})\phi(x)=-\frac{1}{3!}\lambda\phi^{3}\,.
\end{align}
The implication is that while the free theories can be expanded using Fourier theory, interacting theories does not have this property. Hence, interactions must be evaluated by interpreting them as small deviations from the free theory, which is done by a power expansion in a small coupling. We will discuss how this works in \cref{sec:Perturbation theory form path integrals}. 

\subsection*{Dirac Fields}
The next Lagrangian we have use for is the Dirac Lagrangian. This Lagrangian describes fermionic fields, also called \emph{matter fields}. Because of the Pauli exclusion principle, these fields have the property of being anti-commutative, i.e. $\psi(x)\psi(y)=-\psi(y)\psi(x)$. Hence, the square of a fermionic field is zero. From classical quantum mechanics we know that fermionic fields obey the Dirac equation
\begin{align}
    (i\slashed{\partial}-m)\psi=0\,,
\end{align}
where $\psi$ is a complex valued spinor field. The contraction is defined as $\slashed{\partial}=\partial_{\mu}\gamma^{\mu}$, where $\gamma^{\mu}$ are the Dirac gamma matrices, see \cref{sec:Appendix Dirac gamma matrices}. From Lorentz symmetry one can deduce that the correct combination of spinor fields that reproduces the Dirac equation is $\bar{\psi}\psi$, where $\bar{\psi}\equiv\psi
^{\dagger}\gamma^{0}$. Hence, the free Dirac Lagrangian is given by
\begin{align}\label{eq:fermionic LLL}
    \mathcal{L}_{D}=\bar{\psi}(i\slashed{\partial}-m)\psi\,,
\end{align}
where it follows from dimensional analysis that $[\psi]=[\bar{\psi}]=m^{3/2}$. An example of a interaction term is to add a scalar field that interacts with the fermions, known as Yukawa theory
\begin{align}
    \mathcal{L}_{Y}=\mathcal{L}_{D}+\mathcal{L}_{S}-g'\bar{\psi}\psi\phi\,,
\end{align}
where $g'$ is the coupling between fermions and scalars. The most interesting interaction term for fermions are those to vector (gauge) bosons, but we will see an example of such a interaction in the next section.

\subsection*{Vector Fields}
The last example that is relevant for us, is the Lagrangian for vector fields. Vector fields are also bosonic by nature, and the most general free Lagrangian can be written as
\begin{align}
    \mathcal{L}_{V}=\frac{1}{2}\partial_{\mu}A_{\nu}-\frac{1}{2}\partial_{\nu}A_{\mu}-m^{2}A_{\mu}A_{\nu}\,,
\end{align}
where the opposite sign in the kinetic term is to be in accordance with classical electrodynamics. Hence, one can instead write this in terms of the electromagnetic field tensor $F_{\mu\nu}$
\begin{align}\label{eq:vector field LAgrangian}
    \mathcal{L}_{V}=-\frac{1}{4}F_{\mu\nu}F^{\mu\nu}-m^{2}A_{\mu}A_{\nu}\,,
\end{align}
where the minus sign in front of the kinetic term is a convention, and the field tensor is given by
\begin{align}
    F_{\mu\nu}=\partial_{\mu}A_{\nu}-\partial_{\nu}A_{\mu}\,.
\end{align}
There are several possible interactions when introducing vector fields, but we are only interested in interactions with matter fields\footnote{Self-interacting terms are also relevant, but more on such theories in \cref{chap:Geometry of gauge theories}.}. However, let us discuss the mass term for vector fields. In a realistic quantum field theory, the Lagrangian must be invariant under what we call a local phase rotation (or gauge transformation). The mass term is not invariant under such a transformation, and this term must be dropped. 

Let us take the example of a photon coupled to fermions. Instead of writing this down in a heuristic fashion as we did above for Yukawa theory, we can find this term in a more natural way. The Dirac Lagrangian in \cref{eq:fermionic LLL} is not invariant under the transformation
\begin{align}
    \psi(x)\rightarrow e^{i\alpha(x)}\psi\,.
\end{align}
This is a local phase rotation through an \textquote{angle} $\alpha(x)$ that varies arbitrarily from point to point. The difficulty arises from the partial derivative in the Dirac Lagrangian, which brings down $\alpha(x)$ and spoils the invariance. However, by replacing $\partial_{\mu}\rightarrow D_{\mu}$, we can form a gauge invariant Lagrangian
\begin{align}
    \mathcal{L}=\bar{\psi}(i\slashed{D}-m)\psi\,,
\end{align}
where we define the gauge-covariant derivative
\begin{align}
    D_{\mu}\equiv\partial_{\mu}+ieA_{\mu}\,.
\end{align}
With this construction one can add the kinetic term for the photons, giving the well known Lagrangian for Quantum Electrodynamics
\begin{align}
    \mathcal{L}_{QED}=\bar{\psi}(i\slashed{\partial}-m)\psi-\frac{1}{4}F_{\mu\nu}F^{\mu\nu}-e\bar{\psi}\slashed{A}\psi\,,
\end{align}
where $e$ is the electromagnetic coupling.

This example is slightly unmotivated and brief, but this was just meant to introduce the concept of gauge invariance. We will spend a major part of \cref{chap:Geometry of gauge theories} discussing gauge theories and the physical implications that follows. There we will go into more detail why we have to make the replacement $\partial_{\mu}\rightarrow D_{\mu}$ from geometric arguments and not just demand it in a ad-hoc fashion.

Having introduced the different Lagrangians we will encounter in this thesis, we will now move on and discuss quantization of free theories.



\newpage
\section{Canonical Quantization}\label{sec:canonical quantization free theories}
In this section we will briefly discuss the main ideas behind the canonical quantization procedure. Instead of going through all derivations, we will discuss the most important steps and ideas\footnote{For a more elaborate treatment on the canonical quantization procedure, we refer the reader to e.g. \cite{Peskin:257493, sterman_1993}.}.

Let us continue the discussion of a scalar field theory, and in particular a real free scalar field theory. The Lagrangian for a free theory was given in \cref{eq:Lagrangian free scalar theory}, and by applying the Euler-Lagrange equation we obtained the Klein-Gordon equation
\begin{align}
    (\partial^{2}+m^{2})\phi(x)=0\,.
\end{align}

As this is a partial differential equation involving continuous functions, the solution must be interpreted in the sense of distributions\footnote{The mathematics of distributions are necessary when constructing consistent quantum field theories at a formal level.}. For a real scalar field a general Fourier expansion can be written as
\begin{align}
    \phi(\mathbf{x})=\int \frac{d^{3}k}{(2\pi)^{3}}\frac{1}{\sqrt{2E_{\mathbf{k}}}}\,\big(a_{\mathbf{k}}\,e^{i\mathbf{k}\cdot\mathbf{x}}+a^{*}_{\mathbf{k}}\,e^{-i\mathbf{k}\cdot\mathbf{x}}\big)\,,
\end{align}
where $a_{\mathbf{k}}$ and $a_{\mathbf{k}}^{*}$ are Fourier coefficients. To quantize this theory we simply promote the Fourier coefficients to operators $a_{\mathbf{k}}$ and $a_{\mathbf{k}}^{\dagger}$ in Hilbert space. Then $\phi(\mathbf{x})$ becomes a time-independent operator valued distribution, i.e. we are working in the Schrödinger picture. This quantization procedure introduces the equal-time bosonic commutation relations for the operators $a_{\mathbf{k}},a^{\dagger}_{\mathbf{k}}\in\mathcal{H}$
\begin{align}\label{eq:commutaros a and adagger}
    [a_{\mathbf{k}},a_{\mathbf{k'}}^{\dagger}]&=(2\pi)^{3}\delta^{(3)}(\mathbf{k}-\mathbf{k}')\,,
    \\
    [a_{\mathbf{k}},a_{\mathbf{k'}}]&=[a_{\mathbf{k}}^{\dagger},a_{\mathbf{k'}}^{\dagger}]=0\,.
\end{align}
We can now replace the Fourier coefficients with operators and write down the operator valued fields
\begin{align}
    \phi(\mathbf{x})&=\int \frac{d^{3}k}{(2\pi)^{3}}\frac{1}{\sqrt{2E_{\mathbf{k}}}}\,\big(a_{\mathbf{k}}\,e^{i\mathbf{k}\cdot\mathbf{x}}+a^{\dagger}_{\mathbf{k}}\,e^{-i\mathbf{k}\cdot\mathbf{x}}\big)
    \\
    \pi(\mathbf{x})&=\int \frac{d^{3}k}{(2\pi)^{3}}(-i)\sqrt{\frac{E_{\mathbf{k}}}{2}}\big(a_{\mathbf{k}}\,e^{i\mathbf{k}\cdot\mathbf{x}}-a^{\dagger}_{\mathbf{k}}\,e^{-i\mathbf{k}\cdot\mathbf{x}}\big)\,,
\end{align}
where $\pi(\mathbf{x})$ is known as the \emph{conjugated field}.

By performing a Legendre transformation on the Lagrangian the Hamiltonian of the system is given by
\begin{align}
    H&=\int d^{3}x\,\big(\pi(\mathbf{x})\dot{\phi}(\mathbf{x})-\mathcal{L}\big)\nonumber
    \\
    &=\int d^{3}x\big(\frac{1}{2}\pi^{2}+\frac{1}{2}(\nabla\phi)^{2}+\frac{1}{2}m^{2}\phi^{2}\big)\,,
\end{align}
giving the \emph{normal ordered} Hamiltonian
\begin{align}\label{eq:normal ordered}
    H&=\int\frac{d^{3}k}{(2\pi)^{3}}E_{\mathbf{k}}\,a_{\mathbf{k}}^{\dagger}a_{\mathbf{k}}\,,
\end{align}
which follows by using the commutator relations in \cref{eq:commutaros a and adagger}\footnote{For a detailed derivation of this expression, see \cite{Peskin:257493}.}. The normal ordered Hamiltonian in \cref{eq:normal ordered} is an infinite sum of harmonic oscillators with infinitely many degrees of freedom. Hence, we interpret $a_{\mathbf{k}}$ and $a_{\mathbf{k}}^{\dagger}$
as annihilation and creation operators, which introduces the existence of a vacuum state $\ket{0}$, satisfying the condition $a_{\mathbf{k}}\ket{0}=0$.



However, we want our theory to be manifestly Lorentz invariant, so it is useful to transform to the Heisenberg picture
\begin{align}
    \phi(x)=e^{iHt}\phi(\mathbf{x})e^{-iHt}=\int \frac{d^{3}k}{(2\pi)^{3}}\frac{1}{\sqrt{2E_{\mathbf{k}}}}\,\big(a_{\mathbf{k}}\,e^{-ik\cdot x}+a^{\dagger}_{\mathbf{k}}\,e^{ik\cdot x}\big)\,,
\end{align}
where $k^{0}=E_{k}$, and $k^{\mu}x_{\mu}=k\cdot x$. 

Another quantity of interest is to look at the commutator of fields at different spacetime points $[\phi(x),\phi(y)]$. By evaluating this commutator in the following way
\begin{align}
    \bra{0}[\phi(x),\phi(y)]\ket{0}\,,
\end{align}
one finds after choosing a Feynman pole prescription that the propagator in a scalar field theory can be written as
\begin{align}\label{eq:integral rep Fprop}
    D_{F}(x-y)\equiv\int\frac{d^{4}k}{(2\pi)^{2}}\frac{i}{k^{2}-m^{2}+i\epsilon}e^{-ik\cdot(x-y)}\,,
\end{align}
where the poles have been shifted as $k^{0}=\pm(E_{k}-i\epsilon)$, such that they are displaced properly above and below the real axis. This is called the Feynman propagator and can also be written as the time-ordered product
\begin{align}\label{eq:feynman propagator scalar}
    D_{F}(x-y)\equiv\bra{0} \mathcal{T}(\phi(x)\phi(y))\ket{0}\,.
\end{align}

Another way of deriving the Feynman propagator is by using that it is the Green's function of the Klein-Gordon operator, i.e. it must satisfy the following Green's function equation
\begin{align}\label{eq:scalar Feynman propagator equation}
    (\partial^{2}+m^{2})D_{F}(x-y)=-i\delta^{4}(x-y)\,,
\end{align}
where \cref{eq:integral rep Fprop} follows by Fourier expanding $D_{F}$. In all essence the Feynman propagator can be interpreted as the probability amplitude for a \textquote{particle} to travel between two spacetime points. For practical application one often just use the momentum space representation, which is just found by reading of the integrand in \cref{eq:integral rep Fprop}
\begin{align}
    D_{F}(p)=\frac{i}{p^{2}-m^{2}+i\epsilon}\,.
\end{align}
From this discussion we have that the field $\phi(x)$ is the fundamental quantity that permeates all of space, while the particles we observe in nature are excitations of this field. 

\medskip
The same kind of analysis can be made for fermionic fields $\psi(x)$, but with the crucial difference that we have to impose anti-commutation relations. This is not surprising as fermions obey Fermi-Dirac statistics, while scalars obey Bose-Einstein statistics. The Dirac Lagrangian was given in \cref{eq:fermionic LLL}, and by applying the Euler-Lagrange equation the Dirac equation follows
\begin{align}
    (i\slashed{\partial}-m)\psi(x)=0\,,
\end{align}
and if we multiply with $(-i\slashed{\partial}-m)$ from the left, it can be shown that the Dirac equation implies the Klein-Gordon equation. Hence, a free particle solution can be written as a linear combination of plane waves
\begin{align}
    \psi(x)=\sum_{s}u^{s}(p)e^{-ip\cdot x}+v^{s}(p)e^{ip\cdot x}\,,\hspace{1cm}p^{2}=m^{2}\,,
\end{align}
where $u(p)$ and $v(p)$ are interesting objects known as a spinors, while the sum is over spin states. The Fourier expansion for $\psi(x)$ and $\bar{\psi}(x)$ can then be written as
\begin{align}
    \psi(x)&=\int\frac{d^{3}p}{(2\pi)^{3}}\frac{1}{\sqrt{2E_{\mathbf{p}}}}\sum_{s}\Big(a_{\mathbf{p}}^{s}u^{s}(p)e^{-ip\cdot x}+b_{\mathbf{p}}^{s\,\dagger}v^{s}(p)e^{ip\cdot x}\Big)\,,
    \\
    \bar{\psi}(x)&=\int\frac{d^{3}p}{(2\pi)^{3}}\frac{1}{\sqrt{2E_{\mathbf{p}}}}\sum_{s}\Big(a_{\mathbf{p}}^{s\,\dagger}\bar{u}^{s}(p)e^{ip\cdot x}+b_{\mathbf{p}}^{s}\bar{v}^{s}(p)e^{-ip\cdot x}\Big)\,,
\end{align}
where the creation and annihilation operators are not hermitian conjugates of each other, as they were for real scalar fields\footnote{This is because a fermionic field is not real valued and describes two different particles, i.e. a particle and its anti-particle.}. We interpret this expression as $a^{\dagger}$ creates fermions, while $b^{\dagger}$ creates anti-fermions\footnote{The operators $a$ and $b$ will therefore annihilate fermions and anti-fermions respectively.}. 

The next object to define is the Feynman propagator for fermions, which is found by using that it is the Green's function of the Dirac operator
\begin{align}
    (i\slashed{\partial}-m)S_{F}(x-y)=i\delta^{(4)}(x-y)\,,
\end{align}
which by a Fourier transform give the momentum space solution
\begin{align}\label{eq:canocical fermion propagator}
    S_{F}(p)=\frac{i(\slashed{p}+m)}{p^{2}-m^{2}+i\epsilon}\,,
\end{align}
and is often written as the time-ordered product of spinor fields
\begin{align}
    S_{F}(x-y)\equiv\ev{\mathcal{T}(\psi(x)\bar{\psi}(x))}{0}\,.
\end{align}
When we later perform perturbative calculations with so-called Feynman diagrams, we will associate $S_{F}(p)$ with each internal fermion line. 

To end the discussion we will just review some properties of Dirac fields and spinor theory that we will have use for when calculating scattering amplitudes. If a Dirac field acts on a particle momentum state $\ket{p,s}$, we get
\begin{align}\label{eq:spinor relation 1}
    \psi(x)\ket{p,s}_{\text{in}}&=e^{-ip\cdot x}u^{s}(p)\ket{0}\,,
    \\
    _{\text{out}}\bra{p,s}\bar{\psi}(x)&=\bra{0}\bar{u}^{s}(p)e^{ip\cdot x}\,,
\end{align}
and an antiparticle momentum state
\begin{align}\label{eq:spinor relation 2}
    \bar{\psi}(x)\ket{p,s}_{\text{in}}&=e^{-ip\cdot x}\bar{v}^{s}(p)\ket{0}\,,
    \\
    _{\text{out}}\bra{p,s}\bar{\psi}(x)&=\bra{0}v^{s}(p)e^{ip\cdot x}\,,
\end{align}
where all other combinations are zero. When we eventually write down amplitudes from Feynman diagrams, each external fermion leg is represented by a spinor. The exponential factors combine into a delta function ensuring momentum conservation in each vertex. These spinors satisfy the very useful spin sum rules
\begin{align}
    \sum_{s}u^{s}(p)\bar{u}^{s}(p)&=\slashed{p}+m\,,
    \\
    \sum_{s}\bar{v}^{s}(p)v^{s}(p)&=\slashed{p}-m\,,
\end{align}
which is a combination that always appear when calculating scattering amplitudes and simplifies the calculation significantly. 

\medskip
For completeness the most natural thing to do would be to quantize a vector field in this formalism. However, there is another formalism that is much more elegant for treating these types of fields. This formalism is known as the Feynman \emph{Path Integral Formalism}. While the canonical quantization procedure have many natural extensions from regular quantum mechanics, we will eventually generalize to interacting theories and for that treatment the path integral formalism is more suited. Hence, the next sections are devoted to exploring this formalism in more detail.  


\section{The Path Integral Formalism}\label{sec:Path Integral Formalism}
In this section we will consider the path integral formulation of quantum field theory. When we eventually use Feynman diagrams to describe physical processes, we will have the mind-set that the system will take on every configuration possible as it evolves from the intial state to the final state. E.g. photons will split into electrons and positrons which recombines to different photons, leptons (or quarks) annihilate one another and the resulting energy is used to create particles of a different flavour. The point is; anything that can happen will happen. Each of these distinct histories can be thought of as a path through the space of all configurations that describe the state of the system. As previously mentioned the configuration space for a quantum field theory is a Fock space, see \cref{eq:Fock space}. 

But most importantly, each path that the system takes comes with a probabilistic amplitude. The sum over amplitudes associated with each path connecting the initial and final states in the Fock space is interpreted as the probability that some initial state will end up in some final state. So a perturbative expansion using Feynman diagrams is a description of all the possible ways the system can behave. However, the derivation of the path integral in field theory is quite extensive.

But ultimately quantum field theory is grounded in ordinary quantum mechanics; with the essential difference of the number of degrees of freedom \footnote{With the additional requirement that the system has to be invariant under Lorentz transformations, but this is manifest in the path integral formalism.}. Hence, we will instead derive the path integral in quantum mechanics and make the necessary replacements of fields instead of operators. For a more complete treatment and derivation of path integrals in quantum field theory, see e.g. \cite{Peskin:257493,Schwartz:2013pla}. 
 
\subsection{Path Integral in Quantum Mechanics}\label{sec:Path I in QM}
In order to derive the basic idea of the path integral, let us consider the simplest case of one-dimensional quantum mechanics. The time evolution for a state $\ket{\psi(t)}$ can be written as
\begin{align}\label{eq:time evolution of QM state}
    \ket{\psi(t)}=e^{-i(t-t_0)H}\ket{\psi(t_0)}\,,
\end{align}
with the time-independent Hamiltonian 
\begin{align}
    H=\frac{\hat{p}^{2}}{2m}+V(\hat{x})\,,
\end{align}
where $\hat{p}$ and $\hat{x}$ are momentum and position operators acting on states in Hilbert space in the following way
\begin{align}
    \hat{x}\ket{x}&=x\ket{x}\,,
    \\
    \hat{p}\ket{p}&=p\ket{p}\,.
\end{align}

To derive the path integral we will look at an alternative way of describing the time-evolution, using propagators. From the path integral expression of these propagators, we can derive the most important quantity in the path integral formulation, namely Green's functions. To find an expression for the propagator we will consider \cref{eq:time evolution of QM state} and multiply with a position eigenket from the left, giving
\begin{align}
    \psi(x,t)=\braket{x}{\psi(t)}=\bra{x}e^{-i(t-t_0)H}\ket{\psi(t_0)}\,,
\end{align}
and if we insert an identity operator, we can write this as
\begin{align}
    \psi(x,t)&=\int dx_{0}\bra{x}e^{-i(t-t_0)H}\ket{x_0}\braket{x_0}{\psi(t_0)}\nonumber
    \\
    &\equiv\int dx_{0}\,U(x,t;x_0,t_0)\,\psi(x_0,t_0)\,,
\end{align}
where we have defined 
\begin{align}
    U(x,t;x_0,t_0)\equiv\bra{x}e^{-i(t-t_0)H}\ket{x_0}=\braket{x,t}{x_{0},t_{0}}\,,
\end{align}
which is the definiton of the propagator in quantum mechanics. The interpretation of this object goes as follows: we have that $e^{-i(t-t_0)H}\ket{x_0}$ is the state at time $t$, supposed that it was in state $\ket{x_0}$ at time $t_0$. Hence, $\bra{x}e^{-i(t-t_0)H}\ket{x_0}$ is the probability amplitude that the state has evolved to a state $\ket{x}$ at time t, given that it was in the state $\ket{x_0}$ at $t_0$. In other words, say we measure a particle's position at $t_0$ and find it to be $x_0$, then the propagator tells us the probability amplitude for finding the particle at $x$ if we make a new measurent at time $t$.

Let us then see what happens if we split the time interval in two smaller parts, $(t_1-t_0)$ and $(t-t_1)$ and insert the identity
\begin{align}\label{eq:identity for intermediate states}
    1=\int dx_1\,\ket{x_1}\bra{x_1}\,,
\end{align}
giving
\begin{align}\label{eq:split into two time intervals}
    \braket{x,t}{x_{0},t_{0}}&=\bra{x}e^{-i(t-t_1)H}e^{-i(t_1-t_0)H}\ket{x_0}\nonumber
    \\
    &=\int dx_{1}\,\bra{x}e^{-i(t-t_1)H}\ket{x_1}\bra{x_1}e^{-i(t_1-t_0)H}\ket{x_0}\nonumber
    \\
    &=\int dx_{1}\,\braket{x,t}{x_{1},t_{1}}\braket{x_{1},t_{1}}{x_{0},t_{0}}\,.
\end{align}

This expression is identical to the original, only that we have separated it into smaller intermediate pieces. The logic goes as this: we are evolving the system from $x_0$ at time $t_0$ to $x_1$ at $t_1$, then from $x_1$ at $t_1$ to $x$ at $t$, where all positions $x_1$ are integrated over. This must be the same as evolving from $x_0$ to $x$ by passing any arbitrary point $x_1$.


We can then break the problem into $n$ small time intervals $\delta t$ and consider the amplitude for each of those infinitesimal steps. We label the intermediate times $t_j=j\delta t$ where $j=0,1,\dots,n$. With this choice we have that $t_0=0$ and $t_n=t$. If we also insert the identity \cref{eq:identity for intermediate states} for each intermediate step, we get the following expression
\begin{align}\label{eq:iterated matrix element QM}
    \bra{x}e^{-it\hat{H}}\ket{x_{0}}&=\int\prod_{k=1}^{n-1}dx_{k}\,\bra{x}e^{-i\delta t H}\ket{x_{n-1}}\bra{x_{n-1}}\dots\ket{x_2}\bra{x_2}e^{-i\delta t H}\ket{x_1}\bra{x_1}e^{-i\delta t H}\ket{x_{0}}\nonumber
    \\
    &=\int\prod_{k=1}^{n-1}dx_{k}\braket{x_n,t_n}{x_{n-1},t_{n-1}}\braket{x_{n-1},t_{n-1}}{x_{n-2},t_{n-2}}\dots\braket{x_1,t_1}{x_{0},t_{0}}\,.
\end{align}
where we used that $t_n=t$ and $x_n\equiv x$.
Each of these individual propagators can be evaluated by inserting a complete set of momentum eigenstates
\begin{align}\label{eq:}
    \bra{x_{j+1}}e^{-iH\delta t}\ket{x_j}&=\int\frac{dp}{2\pi}\braket{x_{j+1}}{p}\bra{p}e^{-i\big(\frac{\hat{P}^{2}}{2m}+V(\hat{x}_j)\big)\delta t}\ket{x_j}\nonumber
    \\
    &=e^{-iV(x_j)\delta t}\int\frac{dp}{2\pi}e^{-i\frac{p^{2}}{2m}\delta t}e^{ip(x_{j+1}-x_{j})}\,,
\end{align}
where we used that $\braket{p}{x}=e^{-ipx}$. We observe that this is now a Gaussian integral, which has the general solution
\begin{align}
    \int dp\,e^{-ap^{2}+bp}=\sqrt{\frac{\pi}{a}}\,e^{\frac{b^{2}}{4a}}\,.
\end{align}

Using this solution we can write each individual propagator as
\begin{align}\label{eq:intermediate matrix elements QM}
    \bra{x_{j+1}}e^{-iH\delta t}\ket{x_j}&=\sqrt{\frac{m}{2\pi i\delta t}}\,\exp\Big(i\big[\frac{m}{2}\frac{(x_{j+1}-x_j)^{2}}{(\delta t)^{2}}-V(x_j)\big]\delta t\Big)\nonumber
    \\
    &=\mathcal{N}\exp\Big(iL(x_j,\dot{x}_{j})\delta t\Big)\,,
\end{align}
where we defined the integration constant $\mathcal{N}$, which is independent of $x$ and $t$. We also used that the Lagrangian is given by
\begin{align}
    L(x,\dot{x})=\frac{1}{2}m\dot{x}^{2}-V(x)\,,
\end{align}
and by inserting \cref{eq:intermediate matrix elements QM} for each step into \cref{eq:iterated matrix element QM}, we find that
\begin{align}
    \bra{x}e^{-it\hat{H}}\ket{x_{0}}=\mathcal{N}^{n}\int\prod_{k=1}^{n-1}dx_k\,\exp(i\sum_{j=0}^{n-1}\big(\frac{m}{2}\frac{(x_{j+1}-x_j)^{2}}{(\delta t)^{2}}-V(x_j)\big)\delta t)\,.
\end{align}
If we take the limit $n\rightarrow \infty$, and $|t-t_0|$ fixed and finite, the interval $\delta t \rightarrow 0$ and the exponentials will turn into an integral over $dt$. This will lead to the following expression
\begin{align}
    \bra{x}e^{-it\hat{H}}\ket{x_{0}}=\mathcal{N}\int \mathcal{D}x\,e^{i\int dt L(x,\dot{x})}\,,
\end{align}
which is the expression for the path integral in quantum mechanics. We can also write it on the following form
\begin{align}
    \braket{x,t}{x_0,t_0}=\mathcal{N}\int \mathcal{D}x\,e^{iS[x]}\,.
\end{align}
The normalization constant $\mathcal{N}$ has been redefined, and is an infinite constant that will cancel in physical quantities. The measure is defined as
\begin{align}
    \int \mathcal{D}x=\int_{x_0(t_0)=x_0}^{x(t)=x} \mathcal{D}x(t')\equiv \lim_{n\to\infty}\int\prod_{k=1}^{n-1}dx_{k}\,,
\end{align}
where $\mathcal{D}x$ means the sum over all paths $x(t')$ with the given boundary conditions\footnote{This is formally infinite, but we will later see that it does not appear in physical quantities so we will not bother with any formal technicalities.}.

One of the most important features of path integrals is that it allows for an easy computation of matrix elements of operators. These matrix elements are what is called correlators or Green's functions as we call them, and are probably the most important object for any quantum field theory. For example, we saw in \cref{eq:feynman propagator scalar}, that the two-point Green's function is the Feynman propagator, which describes the probability for a particle to travel from a spacetime point to another. 

To see how we can construct these correlators in the path integral formalism, we insert a function $x(t_1)$ and $x(t_2)$ into the path integral. This amount to inserting an operator $\hat{x}(t_1)$ and $\hat{x}(t_2)$, between the initial and final states. We write this as
\begin{align}\label{eq:time ordered product as path integral}
    \bra{x,t}\mathcal{T}(\hat{x}(t_1)\hat{x}(t_2))\ket{x_0,t_0}=\mathcal{N}\int\mathcal{D}x \,x(t_1)x(t_2)\,e^{iS[x]}\,.
\end{align}

In order to show that this the correct form, we start from the right hand side and insert functional delta functions
\begin{align}
    \int\mathcal{D}x \,x(t_1)x(t_2)\,e^{iS[x]}&=\int\mathcal{D}x\mathcal{D}x_1\mathcal{D}x_2\,\delta(x_1-x(t_1))\delta(x_2-x(t_2))x_1x_2\,e^{iS[x]}\,.
\end{align}

For $t_2>t_1$, we can write this as
\begin{align}
    \int\mathcal{D}x \,x(t_1)x(t_2)\,e^{iS[x]}&=\int\mathcal{D}x_1\mathcal{D}x_2\int_{x(t_1)=x_1}^{x(t_2)=x_2}\mathcal{D}x\,x_1x_2\,e^{iS[x]}\nonumber
    \\
    &=\mathcal{N}^{-1}\int\mathcal{D}x_1\mathcal{D}x_2\bra{x}e^{-i(t-t_2)H}x_2\ket{x_2}\bra{x_2}e^{-i(t_2-t_0)H}x_1\ket{x_1}\bra{x_1}e^{-i(t_1-t_0)}\ket{x_0}\nonumber
    \\
    &=\mathcal{N}^{-1}\bra{x}e^{-i(t-t_2)H}\hat{x}e^{-i(t_2-t_1)H}\hat{x}e^{-i(t_1-t_0)H}\ket{x_0}
\end{align}
where we in the last line used the functional spectral decomposition
\begin{align}
    \hat{x}=\int\mathcal{D}x\,x\ket{x}\bra{x}\,.
\end{align}

Now, the relation between the time-dependent operators and time-independent operators are
\begin{align}
    \hat{x}(t)=e^{-itH}\hat{x}e^{itH}\,,
\end{align}
which mean that
\begin{align}
    \mathcal{N}\int\mathcal{D}x \,x(t_1)x(t_2)\,e^{iS[x]}&=\bra{x}e^{-i(t-t_2)H}\hat{x}e^{-i(t_2-t_1)H}\hat{x}e^{-i(t_1-t_0)H}\ket{x_0}\nonumber
    \\
    &=\bra{x}e^{-iHt}\hat{x}(t_2)\hat{x}(t_1)e^{iHt_0}\ket{x_0}\nonumber
    \\
    &=\bra{x,t}\hat{x}(t_2)\hat{x}(t_1)\ket{x_0,t_0}
\end{align}
The exact same result can be found for $t_1>t_2$, giving that in general we write this as in \cref{eq:time ordered product as path integral}. This is because in the path-integral formalism time ordering is automatic. This can also be generalized to $n$ products of operators $\hat{x}$. Furthermore, it can be shown that 
\begin{align}
    \lim_{t_{0},t\to\mp\infty}\bra{x,t}\mathcal{T}(\hat{x}(t_1)\dots\hat{x}(t_n))\ket{x_0,t_0}\propto\bra{0}T\big(\hat{x}(t_1)\dots\hat{x}(t_n)\big)\ket{0}\,.
\end{align}
This is not a trivial result do derive, but it says that at asymptotic times, the initial and final states are the vacuum states of the system. With these consideration we can write what we call $n$-point Green's functions\footnote{Keep in mind that we will sometimes call these correlators, but it stands for the same object.} as
\begin{align}\label{eq:relation path integral and correlators}
     \bra{0}\mathcal{T}(\hat{x}(t_1)\dots\hat{x}(t_n))\ket{0}=\mathcal{N}\int\mathcal{D}x \,x(t_1)\dots x(t_n)\,e^{iS[x]}\,.
\end{align}
Note that the left hand side involve position operators $\hat{x}(t)$, while on the right hand side we have functions $x(t)$. This observation is important to remember when we start talking about fields, and constitutes an important distinction between the canonical and the path integral approach.

In all practicality, the time ordering operator is reduntant in the path integral formulation, but it is common to keep it there. So, when working in the path integral formalism one often use the left hand side of \cref{eq:relation path integral and correlators} for notational simplicity.

\subsection{Path Integrals in Quantum Field Theory}\label{sec:Path I in QFT}
In quantum mechanics we have that position and momentum are promoted to operators, and formulating this in the path integral formalism we have integrals over trajectories in coordinate space. This is often referred to as \emph{first quantization}. In quantum field theory we promote the fields to operators and invoke the canonical commutation relations in \cref{eq:commutaros a and adagger}. This procedure is often referred to as \emph{second quantization}. 

Formulated as a path integral we no longer have an integration over trajectories in coordinate space, but an integration over field configurations. With the work done in the last section, we can move on to the field theoretical description. In \cref{eq:feynman propagator scalar} we defined the Feynman propagator as the two-point Green's function of the Klein-Gordon operator. In general, we can write the $n$-point Green's function as the field correlator
\begin{align}\label{eq:n-point Green's function}
    \mathcal{G}^{n}(x_1,\dots,x_n)=\bra{0}\mathcal{T}(\phi(x_1)\dots\phi(x_n))\ket{0}\,.
\end{align}
By replacing the operators in \cref{eq:relation path integral and correlators} that the path-integral in quantum field theory can be written in terms of the $n$-point Green's function as 
\begin{align}\label{eq:n-point Green's function as path integral}
    \mathcal{G}^{(n)}(x_1,\dots,x_n)=\mathcal{N}\int\mathcal{D}\phi \,\phi(x_1)\dots\phi(x_n)\,e^{iS[\phi]}\,.
\end{align}

It is important to point out that the fields in the path integral are the classical fields, and not the quantum operators. In the canonical quantization formalism, quantization itself lies in the canonical commutation relations by promoting fields to operators. In the path integral formalism we skip the commutation relations and will directly identify the propagator as the Green's function of the differential operator of the theory. This is the main difference of the two approaches, but we will see below that they give the same result.

There is a way to proceed that makes path integral calculations much simpler, and that is to use a \emph{generating gunctional}. The generating functional is defined as the vacuum amplitude in the presence of a source $J(x)$
\begin{align}
    \mathcal{Z}[J]\equiv\braket{0}_{J}=\mathcal{N}\int\mathcal{D}\phi\,\exp\big(iS[\phi]+i\int d^{4}x\,J(x)\phi(x)\big)\,,
\end{align}
where we require the vacuum to be normalized, i.e. $\braket{0}=1$, giving
\begin{align}
    \mathcal{N}^{-1}=\int\mathcal{D}\phi\,\exp\big(iS[\phi]\big)\,.
\end{align}

To go from the generating functional to the Green's function is straigthforward by applying the \emph{functional derivative}\footnote{We will write this as a partial derivative, but it is formally a \emph{variational partial deriative}.} on $\mathcal{Z}[J]$. The effect of the functional derivative is made clear by observing that for
\begin{align}
    J(y)=\int d^{4}x\delta^{4}(x-y)J(x)\,,
\end{align}
we can define
\begin{align}
    \frac{\partial J(x)}{\partial J(y)}=\delta^{4}(x-y)\,,
\end{align}
giving
\begin{align}\label{eq:functional derivative of source and field}
    \frac{\partial}{\partial J(x_1)}\int d^{4}x J(x)\phi(x)=\phi(x_1)\,,
\end{align}
implying that
\begin{align}
    -i\frac{\partial\mathcal{Z}[J]}{\partial J(x_1)}=\mathcal{N}\int\mathcal{D}\phi\,\phi(x_1)\exp\big(iS[\phi]+i\int d^{4}x\,J(x)\phi(x)\big)\,.
\end{align}
The generalization to $n$ functional derivatives evaluated at $J=0$, is given by
\begin{align}\label{eq:GF and GF}
    (-i)^{n}\frac{\partial^{n}\mathcal{Z}[J]}{\partial J(x_1)\cdots\partial J(x_n)}\Big|_{J=0}=\frac{\int\mathcal{D}\phi \,\phi(x_1)\dots\phi(x_n)\,e^{iS[\phi]}}{\int\mathcal{D}\phi \,e^{iS[\phi]}}\,.
\end{align}
Comparing with \cref{eq:n-point Green's function as path integral}, the $n$-point Green's function can be written as
\begin{align}\label{eq:generating functional Z}
    \mathcal{G}^{(n)}(x_1,\dots,x_n)=(-i)^{n}\frac{\partial^{n}\mathcal{Z}[J]}{\partial J(x_1)\cdots\partial J(x_n)}\Big|_{J=0}\,.
\end{align}
In practical calculations this generating functional corresponds to all possible Feynman diagrams, also those that are not describing scattering, i.e. parts of the so-called S-matrix that is trivial\footnote{The S-matrix will be defined in \cref{sec:LSZ and Cross section}.}. Therefore, it is convenient to define a generating functional that corresponds to \emph{connected} Feynman diagrams.

This generating functional for connected diagrams is defined as
\begin{align}
    \mathcal{W}[J]\equiv-i\ln\mathcal{Z}[J]\,,
\end{align}
giving the connected Green's function
\begin{align}\label{eq:generating functional W}
    \mathcal{G}_{c}^{(n)}(x_1,\dots,x_n)=(-i)^{n+1}\frac{\partial^{n}\mathcal{W}[J]}{\partial J(x_1)\cdots\partial J(x_n)}\Big|_{J=0}\,.
\end{align}

This is a simple\footnote{That is not to say it is simple in practice, but in principle.} way of calculating time ordered products. We simply find the generating functional of the theory and perform derivatives to get the time ordered products.

The generating functional is the quantum field theory analog of the partition function in statistical mechanics, where all the information about the theory is encoded in it. That is, if you have an exact closed-form expression for $\mathcal{Z}[J]$, for any particular theory, you have solved it completely. Unfortunately, this is only possible for free theories. For interacting theories we have to use perturbation theory, but in \cref{sec:Perturbation Theory in Quantum Field Theory} we will see a elegant procedure using the Feynman diagram expansion and its connections to the so-called LSZ-reduction formula. 

Before we talk about interacting theories, we would like to show how to solve a quantum field theory exactly. In order to do this, we will first discuss the most important path integral of them all, namely Gaussian path integrals.

\subsection*{Gaussian Path Integrals}
As we have seen Green's functions are given as a path integral weighted with an exponential of the action. The action will always have kinetic terms that are quadratic by nature, i.e. they are Gaussian path integrals. A general Gaussian path integral can be written as
\begin{align}\label{eq:general gaussian path integral}
    \int\mathcal{D}\phi\,e^{-\phi_{x} K_{xy}\phi_{y}}\,,
\end{align}
with the abbreviation
\begin{align}\label{eq:gaussian abbreviation}
    \phi_{x}K_{xy}\phi_{y}=\int d^{4}x\,d^{4}y\,\phi(x)K(x,y)\phi(y)\,,
\end{align}
where $K(x,y)$ is a real symmetric operator. If we expand the fields $\phi$ and $K$ in the same orthonormal basis
\begin{align}
    \phi(x)&=\sum_{i}^{\infty}\xi_{i}\,\phi_{i}(x)
    \\
    K(x,y)&=\sum_{i,j}^{\infty}\phi_{i}(x)K_{ij}\phi_{j}(y)\,,
\end{align}
we can pick out $K_{ij}$ by projecting on the basis functions
\begin{align}
    K_{ij}=\int d^{4}x\,d^{4}y\,\phi_{i}(x)K(x,y)\phi_{j}(y)\,,
\end{align}
giving that we can write \cref{eq:gaussian abbreviation} as
\begin{align}
    \phi_{x}K_{xy}\phi_{y}=\sum_{i,j}\xi_{i}K_{ij}\xi_{j}\,.
\end{align}
and with this rewriting the Gaussian integral takes the form
\begin{align}
    \lim_{n\to\infty}\Big(\prod_{i}^{n}\int d\xi_{i}\Big)\exp[-\sum_{i,j}\xi_{i}K_{ij}\xi_{j}]\,.
\end{align}

To evaluate this integral we use that $K$ is symmetric, and write
\begin{align}
    K_{ij}=\sum_{k,l}O_{ik}A_{kl}O_{lj}\,,
\end{align}
where $O$ is an orthogonal matrix that diagonalizes $K$. The matrix $A$ is diagonal, i.e. $A_{kl}=a_{k}\delta_{kl}$, where $a_k$ are the eigenvalues of $K$. By redefining $x_i=O_{ij}\xi_j$, we get that
\begin{align}
    \Big(\prod_{i}^{n}\int dx_{i}\Big)\exp[-\sum_{i}a_{i}x_{i}^{2}]&=\prod_{i}^{n}\sqrt{\frac{\pi}{a_{i}}}\nonumber
    \\
    &=\sqrt{\pi^{n}}[\det K]^{-1/2}\,,
\end{align}
and the general Gaussian path integral is therefore a product of $n$ Gaussian integrals
\begin{align}
    \int\mathcal{D}\phi\,e^{-\phi_{x} K_{xy}\phi_{y}}=N_{G}[\det K]^{-1/2}\,,
\end{align}
where the normalization constant $N_{G}$ is an infinite constant that will be divided out when calculating Green's functions and $\det K$ is referred to as a \emph{functional determinant}. Functional determinants are interesting and has a wide range of applicability in quantum field theory, but we will not cover this here.

When we want to calculate Green's functions in practice, we will also have terms in the exponential that are linear in the fields. We will then have the following path integral
\begin{align}\label{eq:path integral quadratic and linear in fields}
    \int\mathcal{D}\phi\,e^{-\phi_{x} K_{xy}\phi_{y}+J_{x}\phi_{x}}\,.
\end{align}
The general approach to compute these kinds of integrals is to complete the square. The easiest way to do this is to make the following shift of the fields
\begin{align}\label{eq:shift fields}
    \phi'(x)=\phi(x)-\frac{1}{2}\int d^{4}z\,K^{-1}(x,z)J(z)\,,
\end{align}
with $K^{-1}$ satisfying\footnote{That is, assuming the inverse exist. In general this is not true as we will see when we want to quantize gauge theories.}
\begin{align}\label{eq:normalization Kahler}
    \int d^{4}z\,K(x,z)K^{-1}(z,y)=\delta^{4}(x-y)\,.
\end{align}
We observe that if $K$ is a differential operator, this technique resembles the Green's function method of solving partial differential equations\footnote{Not surprisingly we will later relate $K$ to the differential operator of the theory.}. The path integral is translational invariant, so a shift will not change the integration measure. Inserting this shift into \cref{eq:path integral quadratic and linear in fields}, the exponent takes the form
\begin{align}
    \int d^{4}x\,d^{4}y\,\phi'(x)K(x,y)\phi'(y)-\frac{1}{4}J(x)K^{-1}(x,y)J(y)\,.
\end{align}
The second term is independent on $\phi'$, so this will result in a Gaussian path integral times an exponential function. Since all fields in the path integral formulation are functions, they commute and we can just pull the extra term with source fields $J$ outside the path integral. 

The result after completing the square is then given by
\begin{align}\label{eq:completed the square}
    \int\mathcal{D}\phi\,e^{-\phi_{x} K_{xy}\phi_{y}+J_{x}\phi_{x}}=e^{\frac{1}{4}J_{x}K_{xy}^{-1}J_{y}}\int\mathcal{D}\phi\,e^{-\phi_{x} K_{xy}\phi_{y}}\,.
\end{align}
This is now on a form where it is easier to see the action of functional derivaties on the path integral. From \cref{eq:functional derivative of source and field} we know that by acting with functional derivatives on a path integral, we \textquote{pull} down a field from the exponent, i.e.
\begin{align}
    \frac{\partial^{2}}{\partial J(x_1)\partial J(x_2)}\int\mathcal{D}\phi e^{-\phi_{x}K_{xy}\phi_{y}+J_{x}\phi_{x}}\big|_{J=0}=\int\mathcal{D}\phi\,\phi(x_1)\phi(x_2)e^{-\phi_{x}K_{xy}\phi_{y}}\,,
\end{align}
but since we have completed the square in \cref{eq:completed the square}, all the $J$ dependence is in the factor in front of the path integral, so we can also write
\begin{align}
    \int\mathcal{D}\phi\,\phi(x_1)\phi(x_2)e^{-\phi_{x}K_{xy}\phi_{y}}=\Big[\frac{\partial^{2}}{\partial J(x_1)\partial J(x_2)}e^{\frac{1}{4}J_{x}K_{xy}^{-1}J_{y}}\Big]_{J=0}\int\mathcal{D}\phi e^{-\phi_{x}K_{xy}\phi_{y}}\,.
\end{align}

Calculating the bracket yields
\begin{align}
    \frac{\partial^{2}}{\partial J(x_1)\partial J(x_2)}e^{\frac{1}{4}J_{x}K_{xy}^{-1}J_{y}}\Big|_{J=0}=\frac{1}{2}K^{-1}(x_1,x_2)\,,
\end{align}
giving the two-point Gaussian path integral
\begin{align}\label{eq:two-point gaussian}
    \frac{1}{2}K^{-1}(x_1,x_2)=\frac{\int\mathcal{D}\phi\,\phi(x_1)\phi(x_2)e^{-\phi_{x}K_{xy}\phi_{y}}}{\int\mathcal{D}\phi e^{-\phi_{x}K_{xy}\phi_{y}}}\,,
\end{align}
and generalizing this to $n$-point Gaussian path integrals is straightforward
\begin{align}
    \Big[\frac{\partial^{n}}{\partial J(x_1)\cdots\partial J(x_n)}e^{\frac{1}{4}J_{x}K_{xy}^{-1}J_{y}}\Big]_{J=0}=\frac{\int\mathcal{D}\phi\,\phi(x_1)\cdots\phi(x_n)e^{-\phi_{x}K_{xy}\phi_{y}}}{\int\mathcal{D}\phi e^{-\phi_{x}K_{xy}\phi_{y}}}\,.
\end{align}
This looks very much like the expression we found for the generating functional \cref{eq:generating functional Z}, and will be helpful after the action has been specified.

\subsection{Quantization of Scalar Field Theory}
Let us then move on and finally try to quantize a theory in this formalism. The scalar theory is easiest so we show the main steps, which we will use when we come to quantization of fermionic fields and vector fields.

The action for a free real scalar theory follows from \cref{eq:Lagrangian free scalar theory}
\begin{align}
    S_{0}[\phi]=\int d^{4}x\,\Big(\frac{1}{2}\partial_{\mu}\phi\partial^{\mu}\phi-\frac{1}{2}m^{2}\phi^{2}\Big)\,,
\end{align}
and the generating functional can be written as
\begin{align}\label{eq:generating functional scalar theory}
    \mathcal{Z}_{0}[J]=\mathcal{N}_{0}\int\mathcal{D}\phi\,\exp(iS_{0}[\phi]+i\int d^{4}x\,J(x)\phi(x))\,.
\end{align}
We would like to rewrite the action on a form such that we can use the results we found for Gaussian path integrals. To do this, we use partial integration and rewrite the action as
\begin{align}\label{eq:rewritten scalar A}
    S_{0}[\phi]=-\frac{1}{2}\int d^{4}x\,d^{4}y\,\phi(x)\delta^{(4)}(x-y)(\partial^{2}+m^{2})\phi(y)\,.
\end{align}
If we insert this expression into \cref{eq:generating functional scalar theory} and compare with the Gaussian path integral \cref{eq:path integral quadratic and linear in fields}, we observe that
\begin{align}\label{eq:kahler scalar}
    K(x,y)=\frac{i}{2}\delta^{(4)}(x-y)(\partial^{2}+m^{2})\,,
\end{align}
and from \cref{eq:normalization Kahler} it follows that the inverse must satisfy
\begin{align}\label{eq:blala}
    \frac{1}{2}(\partial^{2}+m^{2})K^{-1}(x,y)=-i\delta^{(4)}(x-y)\,.
\end{align}
If this step is obscure, just insert \cref{eq:kahler scalar} into \cref{eq:normalization Kahler}, giving the above equation. We observe that \cref{eq:blala} is a Green's function equation for the Klein-Gordon operator. The recipe of quantization in the path integral formalism is then by defining the propagator as
\begin{align}\label{eq:Feynman propagator identity}
    D_{F}(x,y)\equiv\frac{1}{2}K^{-1}(x,y)\,,
\end{align}
which reproduces Klein-Gordon propagator equation \cref{eq:scalar Feynman propagator equation} in canonical quantization
\begin{align}
    (\partial^{2}+m^{2})D_{F}(x,y)=-i\delta^{(4)}(x-y)\,,
\end{align}
which after a Fourier transform results in the coordinate space propagator given in \cref{eq:integral rep Fprop}. This identification shows that the path integral formalism and the canonical quantization procedure give the same result. This might seem backwards and a lot of work for nothing new, but it shows that the Feynman propagator follows from a general treatment of the path integral.

With the identification \cref{eq:Feynman propagator identity} the generating functional for a free scalar theory follows by completing the square of \cref{eq:generating functional scalar theory}, giving
\begin{align}\label{eq:rewritten generating functional scalar theory}
    \mathcal{Z}_{0}[J]=\exp(-\frac{1}{2}\int\,d^{4}x\,d^{4}y\,J(x)D_{F}(x,y)J(y))\,.
\end{align}
This is an exact result, and from it we can learn all there is about the theory.

Then we can also see that by performing the following functional derivatives on \cref{eq:rewritten generating functional scalar theory}, will give
\begin{align}
    \mathcal{G}^{(2)}(x,y)=-\frac{\partial^{2}\mathcal{Z}_{0}[J]}{\partial J(x)\partial J(y)}\big|_{J=0}=D_{F}(x,y)\,,
\end{align}
and as a path integral it follows from \cref{eq:relation path integral and correlators} that
\begin{align}
    D_{F}(x,y)=\frac{\int\mathcal{D}\phi\,\phi(x)\phi(y)e^{iS_{0}[\phi]}}{\int\mathcal{D}\phi \,e^{iS_{0}[\phi]}}\,.
\end{align}

We already know that the two-point function is the Feynman propagator, but we wanted to show that there are many different ways of representing Green's functions in the path integral formalism.

The most important point here is that the quantization procedure did not involve promoting the fields to operators or invoking commutation relations, but instead by imposing the identity in \cref{eq:Feynman propagator identity}.

\subsection{Quantization of Fermionic Fields}\label{sec:quantization of fermion}
To quantize fermionic theories the general features of the quantization procedure is the same as for scalar theory. The main difference is that one have to use Grassmann calculus as fermionic fields are anti-commuting. It would take to much space to fully cover Grassmann numbers and Gaussian path integrals over Grassmannian fields, so we will instead cover the basic results for Green's functions in fermionic theory. For more detail on Grassmanian path integrals we refer the reader to \cite{Peskin:257493}.

The generating functional can be written in a similar way as for scalar theory, only that we need two Grassmann valued sources $\bar{\eta}$ and $\eta$. In general, one has that the Gaussian path integral over Grassmannian fields is given by
\begin{align}
    \int\mathcal{D}\bar{\psi}\mathcal{D}\psi\,e^{-\bar{\psi}_{x}K_{xy}\psi_{y}}=\det K\,,
\end{align}
and the result after completing the square
\begin{align}
    \int\mathcal{D}\bar{\psi}\mathcal{D}\psi\,e^{-\bar{\psi}_{x}K_{xy}\psi_{y}+\bar{\eta}_{x}\psi_{x}+\bar{\psi}_{x}\eta_{x}}=\det K\,e^{\bar{\eta}_{x}K_{xy}^{-1}\eta}\,.
\end{align}
The free Dirac generating functional can then be written as
\begin{align}\label{eq:free Dirac generating functional}
    \mathcal{Z}_{0}[\bar{\eta},\eta]=\mathcal{N}\int\mathcal{D}\bar{\psi}\mathcal{D}\psi\exp(iS_{0}[\bar{\psi},\psi]+i\int d^{4}x\,\bar{\eta}(x)\psi(x)+\bar{\psi}(x)\eta(x))\,,
\end{align}
with the free Dirac action
\begin{align}
    S_{0}[\bar{\psi},\psi]=\int d^{4}x\,\bar{\psi}(i\slashed{\partial}-m)\psi\,,
\end{align}
and if we do the same as for the scalar theory in \cref{eq:rewritten scalar A}, it follows that
\begin{align}
    K(x,y)=-i\delta^{(4)}(x-y)(i\slashed{\partial}-m)\,,
\end{align}
satisfying
\begin{align}
    \int d^{4}z\,K(x,z)K^{-1}(y,z)=\delta^{(4)}(x-y)\,,
\end{align}
which implies that the inverse must satisfy the Green's function equation for the Dirac operator, i.e.
\begin{align}
    (i\slashed{\partial}-m)K^{-1}(x,y)=i\delta^{(4)}(x-y)\,.
\end{align}

Everything is similar to the scalar theory, so we identity the inverse as the Green's function of the Dirac operator
\begin{align}
    S_{F}(x,y)\equiv K^{-1}(x,y)\,,
\end{align}
where again the solution is obtained by a Fourier transform of the Green's function equation, giving the momentum space solution
\begin{align}
    S_{F}(x,y)=\frac{i}{\slashed{p}-m+i\epsilon}\,,
\end{align}
reproducing the same propagator given in \cref{eq:canocical fermion propagator} defined in canonical quantization.

For completeness, let us complete the square by shifting the fields in \cref{eq:free Dirac generating functional}, giving\footnote{See \cref{eq:shift fields} how to shift the fields.}
\begin{align}
    \mathcal{Z}_{0}[\bar{\eta},\eta]=\exp(-\int d^{4}x\,d^{4}y\,\bar{\eta}(x)S_{F}(x,y)\eta(y))\,.
\end{align}
Applying functional derivatives to this we can find $n$-point Green's functions, which as path integrarls can be written as
\begin{align}
    \mathcal{G}^{(n)}(x_1,\dots x_n;y_1,\dots y_n)&=\bra{0}\mathcal{T}(\psi(x_1)\dots\psi(x_n)\bar{\psi}(y_1)\dots\bar{\psi}(y_n))\ket{0}\nonumber
    \\
    &=\frac{\int\mathcal{D}\bar{\psi}\mathcal{D}\,\psi\,\psi(x_1)\dots\psi(x_n)\bar{\psi}(y_1)\dots\bar{\psi}(y_n)\,e^{iS_{0}[\bar{\psi},\psi]}}{\int\mathcal{D}\bar{\psi}\mathcal{D}\psi\,e^{iS_{0}[\bar{\psi},\psi]}}\,.
\end{align}

With these constructions one have all information needed to specify the whole theory, but this is not very interesting as the fields are non-interacting.
After we have connected Green's functions to the $S$-matrix and subsequently to scattering amplitudes, we have all we need to compute physical processes using Feynman diagrams. 



\subsection{Quantization of Abelian Gauge Fields}\label{sec:quantization of abelian}
Lastly, in this subsection we will investigate how to quantize an Abelian gauge theory, or Maxwell theory, as we are ultimately interested in describing the dynamics of the electromagnetic field. The quantization of non-Abelian gauge theories will be made after we have derived the Yang-Mills Lagrangian from a geometric viewpoint in \cref{chap:Geometry of gauge theories}. But the basic set up is the same for both Abelian and non-Abelian theories, so it is useful to do it in detail for the easiest case and point out the differences in the non-Abelian case.

The action for a Maxwell theory is given by\footnote{This follows from the vector field Lagrangian in \cref{eq:vector field LAgrangian}.}
\begin{align}
    S[A]=-\frac{1}{4}\int d^{4}x\,F_{\mu\nu}F^{\mu\nu}\,,
\end{align}
and as usual we want to find the two-point Green's function for this theory. The relevant path integral for this theory is
\begin{align}\label{eq:Maxwell path integral}
    \int\mathcal{D}A\,e^{iS[A]}\,,
\end{align}
where 
\begin{align}
    \int\mathcal{D}A=\int\mathcal{D}A^{0}\int\mathcal{D}A^{1}\int\mathcal{D}A^{2}\int\mathcal{D}A^{3}\,,
\end{align}
since Abelian fields commute we have one scalar path integral per component.  

Let us use the prescription we have used for both the scalar field and Dirac field quantization, i.e. use the Gaussian path integral to find $K^{-1}$ and relate it to the propagator. To identify the differential operator we integrate by parts and use a delta function to write
\begin{align}
    S[A]&=\frac{1}{2}\int d^{4}x\,d^{4}y\,A_{\mu}(x)\delta^{(4)}(x-y)(\partial^{2}g^{\mu\nu}-\partial^{\mu}\partial^{\nu})A_{\nu}(y)\,,
\end{align}
from which we identity
\begin{align}
    K^{\mu\nu}(x,y)=-\frac{i}{2}\delta^{(4)}(x-y)(\partial^{2}g^{\mu\nu}-\partial^{\mu}\partial^{\nu})\,.
\end{align}

Let us continue in the naive way and assume the inverse exist, giving the Green's function equation
\begin{align}
    \frac{1}{2}(\partial^{2}g^{\mu\nu}-\partial^{\mu}\partial^{\nu})K_{\nu\rho}^{-1}(x,y)=i\delta_{\rho}^{\mu}\delta^{(4)}(x-y)\,,
\end{align}
or in terms of the Fourier transform
\begin{align}
    \frac{1}{2}(-k^{2}g^{\mu\nu}+k^{\mu}k^{\nu})K_{\nu\rho}^{-1}(k)=i\delta_{\rho}^{\mu}\,.
\end{align}
However, this is not a well defined equation since $(-k^{2}g^{\mu\nu}+k^{\mu}k^{\nu})$ is singular, i.e. it has an eigenvector $k_{\nu}$ with eigenvalue zero, resulting in that $K^{-1}$ has no solution.

The difficulty is due to the gauge freedom of the Maxwell action, i.e. the Maxwell action is invariant under the transformation
\begin{align}
    A_{\mu}(x)\rightarrow A_{\mu}(x)+\partial_{\mu}\alpha(x)\,,
\end{align}
where the troublesome modes are those for which $A_{\mu}(x)=\partial_{\mu}\alpha(x)$, i.e. those that are gauge-equivalent to $A_{\mu}(x)=0$. As a result, the path integral is badly divergent, as we are redundantly integrating over an infinite set of equivalent fields.

To fix the problem there is a trick due to Fadeev and Popov \cite{Faddeev:1967fc}, where we can isolate the interesting part of the functional integral. To this end, we define a functional $G[A]$ that we wish to set equal to zero as a gauge-fixing condition. There are a wide range of possibilities, but the ones we will focus on here is 
\begin{align}
    \text{Lorentz gauge:}\hspace{1cm}G[A]&=\partial_{\mu}A^{\mu}\,,
    \\
    \text{Axial gauge:}\hspace{1cm}G[A]&=n_{\mu}A^{\mu}\,.
\end{align}

For the Abelian case we will only consider Lorentz gauge, but for the non-Abelian case we will take a closer look at axial gauges. 

We can constrain the path integral to cover only the configurations with $G[A]=0$ by inserting a functional delta function. However, we can not blindly put this in as it would change the path integral. Instead, we can insert an identity by using the functional identity
\begin{align}\label{eq:functional identity}
    \int\mathcal{D}\alpha\,\delta(G[A^{\alpha}])=\det\Big(\frac{\partial G[A^{\alpha}]}{\partial\alpha}\Big)^{-1}\,,
\end{align}
or
\begin{align}
    1=\det(\frac{\partial G[A^{\alpha}]}{\partial\alpha})\int\mathcal{D}\alpha\,\delta(G[A^{\alpha}])\,,
\end{align}
where $A^{\alpha}$ is the gauge-transformed field. Since the gauge transformation is just a shift, we have that $S[A]=S[A^{\alpha}]$ and $\mathcal{D}A=\mathcal{D}A^{\alpha}$. Hence, we can write the path integral as
\begin{align}
    \int\mathcal{D}\alpha\int\mathcal{D}A^{\alpha}e^{iS[A^{\alpha}]}\,\delta(G[A^{\alpha}])\det(\frac{\partial G[A^{\alpha}]}{\partial\alpha})\,.
\end{align}
Now, $A^{\alpha}$ is just a dummy integration variable, so we can just rename it back to $A$. Also, for Abelian fields we have that the functional determinant is independent of $A$. This is easily seen by choosing Lorentz gauge $G[A^{\alpha}]=\partial^{\mu}A_{\mu}+\partial^{2}\alpha$, giving
\begin{align}
    \det(\frac{\partial G[A^{\alpha}]}{\partial\alpha})&=\det(\partial^{2})\,.
\end{align}
Hence, we can pull it out of the path integral and obtain
\begin{align}
    \int\mathcal{D}A\,e^{iS[A]}=\det(\frac{\partial G[A^{\alpha}]}{\partial\alpha})\int\mathcal{D}\alpha\int\mathcal{D}A\,e^{iS[A]}\,\delta(G[A])\,,
\end{align}
where we kept the $\alpha$-dependence in the functional determinant as it is gauge-invariant and is unimportant as it will cancel for Green's functions. The integration over $\alpha$ is an infinite constant, but this will also cancel so we will not concern ourselves with it. The effect of rewriting the path integral in this fashion is that we have due to the delta function restricted ourselves to physical inequivalent field configurations. 

Let us then specify the gauge fixing functional to be the class of Lorentz gauges 
\begin{align}
    G[A]=\partial^{\mu}A_{\mu}-\omega(x)\,,
\end{align}
where $\omega(x)$ is a scalar function. The functional determinant does not change, so we get
\begin{align}
    \int\mathcal{D}A\,e^{iS[A]}&=\Big(\det(\partial^{2})\int\mathcal{D}\alpha\Big)\int\mathcal{D}A\,e^{iS[A]}\,\delta(\partial^{\mu}A_{\mu}-\omega(x))\nonumber
    \\
    &=N(\alpha)\int\mathcal{D}A\,e^{iS[A]}\,\delta(\partial^{\mu}A_{\mu}-\omega(x))\,.
\end{align}

Since $\omega$ is unspecified, this equation remains valid if we replace the right hand side with any linear combination of $\omega$. The final trick is then to add such a linear combination by adding 1 again. To do this we consider the following Gaussian integral
\begin{align}
    \int\mathcal{D}\omega\,\exp(-i\int d^{4}x\,\frac{\omega^{2}}{2\xi})=\lim_{n\to\infty}\sqrt{(2i\xi\pi)^{n}}\equiv N^{-1}(\xi)\,,
\end{align}
which mean that can then safely add the following factor to the path integral
\begin{align}\label{eq:gaussian weigh in quantization}
    N(\xi)\int\mathcal{D}\omega\,\exp(-i\int d^{4}x\,\frac{\omega^{2}}{2\xi})=1\,,
\end{align}
giving
\begin{align}
    \int\mathcal{D}A\,e^{iS[A]}=\Tilde{N}\int\mathcal{D}\omega\int\mathcal{D}A\,\delta(\partial^{\mu}A_{\mu}-\omega(x))\,\exp(iS[A]-i\int d^{4}x\,\frac{\omega^{2}}{2\xi})\,,
\end{align}
where $\Tilde{N}=N(\xi)N(\alpha)$ is still just an infinite constant that is unimportant. The above insertion corresponds to a Gaussian weighting function that is normalized to one, centered around $\omega=0$, and integrated over all possible $\omega$. This is the same as the continous limit of making all possible linear combinations of $\omega$, just as we wanted. Also, we can choose $\xi$ to be any finite constant and is referred to as a gauge fixing parameter.

Let us then use the delta function to perform the integration over $\omega$, giving
\begin{align}
    \int\mathcal{D}A\,e^{iS[A]}=\Tilde{N}\int\mathcal{D}A\,\exp(iS[A]-i\int d^{4}x\,\frac{1}{2\xi}(\partial^{\mu}A_{\mu})^{2})\,.
\end{align}
Effectively, we have added a term to the action that ensures that the previously divergent path integral has a nice behaviour. 

We can then write the gauge-fixed action as
\begin{align}\label{eq:Abelian gauge fixed action}
    S_{GF}[A]=\frac{1}{2}\int d^{4}x\,d^{4}y\,A_{\mu}(x)\delta^{(4)}(x-y)\big(\partial^{2}g^{\mu\nu}-\Big(1-\frac{1}{\xi}\Big)\partial^{\mu}\partial^{\nu}\big)A_{\nu}(y)\,,
\end{align}
and it follows that the Green's function equation is given by
\begin{align}\label{eq:photon propagator equation}
    \frac{1}{2}\big(\partial^{2}g^{\mu\nu}-\Big(1-\frac{1}{\xi}\Big)\partial^{\mu}\partial^{\nu}\big)K_{\nu\rho}^{-1}(x,y)=i\delta_{\rho}^{\mu}\delta^{(4)}(x-y)\,.
\end{align}
which is non-singular, i.e. the propagator exist. Thus, we define the Abelian gauge field (photon) propagator as
\begin{align}
    D_{F}^{\mu\nu}(x,y)\equiv\frac{1}{2}K^{-1}(x,y)\,.
\end{align}

To find the explicit solution to the propagator equation we use that the most general second rank symmetric tensor can be expanded as (in momentum space)
\begin{align}\label{eq:parametrization of propagator}
    D_{F}^{\mu\nu}(k)=Ag^{\mu\nu}+Bk^{\mu}k^{\nu}\,,
\end{align}
which inserted into \cref{eq:photon propagator equation} will give $A=-i/k^{2}$ and $B=-i(1-\xi)/k^{4}$, and the the momentum space propagator takes the form
\begin{align}\label{eq:photon propagator without gauge choice}
    D_{F}^{\mu\nu}(k)=\frac{-i}{k^{2}+i\epsilon}\Big(g^{\mu\nu}-(1-\xi)\frac{k^{\mu}k^{\nu}}{k^{2}}\Big)\,.
\end{align}

To actually use this expression in calculations we have to specify a gauge choice for $\xi$, and the most common is the Feynman gauge $\xi=1$. Other choices are possible as well, e.g. the Landau gauge $\xi=0$ or the Yennie gauge $\xi=3$.

Finally, we can show that the infinite constants $\mathcal{N}(\xi)\mathcal{N}(\alpha)$ we have in the path integral vanishes for physical quantities, i.e. we have that the $n$-point Green's function takes the form
\begin{align}
    \bra{0}\mathcal{T}\,F[A]\ket{0}=\frac{\int\mathcal{D}A\,F[A]\,\exp(iS[A]-i\int d^{4}x\,\frac{1}{2\xi}(\partial^{\mu}A_{\mu})^{2})}{\int\mathcal{D}A\,\exp(iS[A]-i\int d^{4}x\,\frac{1}{2\xi}(\partial^{\mu}A_{\mu})^{2})}\,,
\end{align}
where $F[A]$ is a collection of fields.

This leads us to the end of the discussion regarding free fields. We have now explored all the relevant concepts we will use in the transition to interacting theories using the the path integral formalism.