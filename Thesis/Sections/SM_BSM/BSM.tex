\section{Proposal model}
Parity symmetry is suggested to be restored at high energies by the introduction of 
a right handed weak symmetry, leading to right handed weak charged bosons, $W_R^{\pm}$. 
These mediators, as with the rest, decays faster than the detection ability at CERN, thus 
evidence of such a boson would come from the detection of a heavy neutrino. Heavy neutrinos can 
be produced in proton proton collisions through either the right handed $W_R^{\pm}$ bosons or
the SM $W^{\pm}$ bosons. 


\begin{figure}[h!]
    \centering
    \caption{Proton-proton collision with heavy neutrino production via SM $W^{\pm}$ boson into 3 lepton final state. Read the graph from left to right.}
\begin{tikzpicture}
    \begin{feynman}
    
        \vertex at (0.5, 1.) (a2){\(q\)} ;
        \vertex at (0.5, -1.)  (a5){\(\bar{q}'\)} ;

        \vertex at (1.5, 0.0) (c);
        \vertex at (3., 0.0) (d);

        \vertex at (5.5, 1.) (f1) {\(l^{\pm}\)} ;
        \vertex at (3.8,-0.88) (f2) {\(N\)};

        \vertex at (5.5, -0.2) (f3) {\(l\)} ;

        \vertex at (5, -1.7) (e)  ;

        \vertex at (6, -1.0) (f4) {\(l\)} ;
        \vertex at (6, -2.2) (f5) {\(\nu\)} ;

        
    \diagram*{


        (a2) -- [fermion] (c) -- [boson] (d),
    
        (a5) -- [anti fermion] (c) -- [boson, edge label = {$W^{\pm}$}] (d),
        (d) -- [fermion] (f1),
        (d) -- [fermion] (f2),
        (f2) -- [fermion] (f3),
        (f2) -- [boson, edge label = {$W^{(*)}$}] (e),
        (e) -- [fermion] (f4),
        (e) -- [fermion] (f5),
        ;
    };
    \end{feynman}
    \end{tikzpicture}
    \label{fig:Target_model_1}
    
\end{figure}



\begin{figure}[h!]
    \centering
    \caption{Proton-proton collision with heavy neutrino production via right-handed $W_R^{\pm}$ boson into 3 lepton final state. Read the graph from left to right.}

\begin{tikzpicture}
    \begin{feynman}
    
        \vertex at (0.5, 1.) (a2){\(q\)} ;
        \vertex at (0.5, -1.)  (a5){\(\bar{q}'\)} ;

        \vertex at (1.5, 0.0) (c);
        \vertex at (3., 0.0) (d);

        \vertex at (5.5, 1.) (f1) {\(l^{\pm}\)} ;
        \vertex at (3.8,-0.88) (f2) {\(N\)};

        \vertex at (5.5, -0.2) (f3) {\(l\)} ;

        \vertex at (5, -1.7) (e)  ;

        \vertex at (6, -1.0) (f4) {\(l\)} ;
        \vertex at (6, -2.2) (f5) {\(\nu\)} ;

        
    \diagram*{


        (a2) -- [fermion] (c) -- [boson] (d),
    
        (a5) -- [anti fermion] (c) -- [boson, edge label = {$W_R^{\pm}$}] (d),
        (d) -- [fermion] (f1),
        (d) -- [fermion] (f2),
        (f2) -- [fermion] (f3),
        (f2) -- [boson, edge label = {$W_R^{(*)}$}] (e),
        (e) -- [fermion] (f4),
        (e) -- [fermion] (f5),
        ;
    };
    \end{feynman}
    \end{tikzpicture}
    \label{fig:Target_model_2}
    \end{figure}

Note that the right most $W$ and $W_R$ bosons have an asterix as a superfix. This is because 
the bosons can be virtual. This means that if the mass of the heavy neutrino is less than the W boson,
it cannot produce it, but it can produce a virtual one such that the decay still happens. 