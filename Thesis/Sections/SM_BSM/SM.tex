
\begin{figure}[H]
    \includegraphics[width=\linewidth]{Figures/SM/Standard_Model_of_Elementary_Particles.svg.png}
    \caption[The Standard Model]{The SM of elementary particles. Source \href{https://upload.wikimedia.org/wikipedia/commons/thumb/0/00/Standard_Model_of_Elementary_Particles.svg/1200px-Standard_Model_of_Elementary_Particles.svg.png}{here}. Accessed 07.10.22}
    \label{fig:smdiagram}
\end{figure}

\section{Structure and composition of the Standard Model}
This section will describe the SM in a phenomelogical way, as the mathematics and 
physical reasoning behind the theory is not of great 
importance to understand the work, results or discussion in this thesis. For a more 
technical explanation, see (Pich, 2008)\cite{Pich:819632} for a 
well written paper containing some more SM fundamentals as well as summarizing the 
experimental status regarding the SM.
For more mathematical understanding of the SM, Peskin and Schroeder's "An introduction 
to Quantum Field theory" (Peskin and Schroeder, 1995)\cite{Peskin:1995ev}
 is highly recommended. Finally, see Thomson's "Modern Particle physics" (Thomson, 2013)
 \cite{Thomson:2013zua} for a very comprehensive and up-to-date book that is easy to read and 
 understand. \par
The SM is to physicists what the periodic table is to chemists, and is to this day the most fundamental description of 
matter as we know it at the subatomic scale. It comprises two parent class particles, fermions which have half integer 
spin and bosons which have full integer spin. The fermions are comprised of quarks and leptons. The model contains 
6 leptons, 6 quarks coming in 3 colors each, 4 gauge bosons mediating the electroweak interactions $(\gamma, W^{\pm}, Z^{0})$, 
8 gluons behind the strong interaction and one 
scalar Higgs explaining particle masses, all of which are shown in figure \ref{fig:smdiagram}.



\subsection*{Fermions}
The fermions are the building blocks of matter, and contain two types of particles, leptons and quarks. The up and down quarks form protons and neutrons, which together with electrons form the atoms.
Fermions, unlike bosons, are spin half particles. The fermions are grouped into three so-called families:
\begin{equation*}
    \begin{bmatrix}
        \nu_e & u \\
        e^{-} & d^{\, '} 
    \end{bmatrix},\quad
    \begin{bmatrix}
        \nu_{\mu} & c \\
        \mu^{-} & s^{\, '}
    \end{bmatrix},\quad
    \begin{bmatrix}
        \nu_{\tau} & t \\
        \tau^{-} & b^{\, '}
    \end{bmatrix}
\end{equation*}
The left column contains the leptons whilst the right column contains the quarks. Within the left column, the subscripted $\nu$ denotes what kind of neutrino that corresponds 
to the given lepton. Here, the first family consists of the electron, the electron neutrino, the up and down quarks. The second family consists of the muon and the
muon neutrino, the charm and strange quarks. The third family consists of the tau and the tau neutrino, the top and bottom quarks. The masses of these particles 
increases for each particle in the matrix as the family number increases, i.e. the muon is heavier than the electron, and the tau is heavier than the muon, and so 
on for the other charged fermions. It is not known wether or not the neutrinos follow this pattern as well due to their very low mass. Below is a table with specific properties of the fermions.\par

% Please add the following required packages to your document preamble:
% \usepackage{multirow}
\begin{table}[H]
    
    \begin{tabular}{|ccccccc|}
    \hline
    \multicolumn{1}{|l|}{\textbf{Generation}}         & \multicolumn{1}{l|}{\textbf{Name}}     & \multicolumn{1}{l|}{\textbf{Symbol}} & \multicolumn{1}{l|}{\textbf{Antiparticle}} & \multicolumn{1}{l|}{\textbf{Spin}} & \multicolumn{1}{l|}{\textbf{Charge}} & \multicolumn{1}{l|}{\textbf{Mass (MeV$/c^2$)}} \\ \hline
    \multicolumn{7}{|c|}{\textbf{Quarks}}                                                                                                                                                                                                                                                                       \\ \hline
    \multicolumn{1}{|c|}{\multirow{2}{*}{\textbf{1}}}                  & \multicolumn{1}{c|}{up}                & \multicolumn{1}{c|}{u}               & \multicolumn{1}{c|}{$\bar{u}$}             & \multicolumn{1}{c|}{$1/2$} & \multicolumn{1}{c|}{$2/3$}   & $2.2_{-0.4}^{+0.6}$                            \\ \cline{2-7} 
    \multicolumn{1}{|c|}{}                   & \multicolumn{1}{c|}{down}              & \multicolumn{1}{c|}{d}               & \multicolumn{1}{c|}{$\bar{d}$}             & \multicolumn{1}{c|}{$1/2$} & \multicolumn{1}{c|}{$-1/3$}  & $4.6_{-0.4}^{+0.5}$                            \\ \hline
    \multicolumn{1}{|c|}{\multirow{2}{*}{\textbf{2}}} & \multicolumn{1}{c|}{charm}             & \multicolumn{1}{c|}{c}               & \multicolumn{1}{c|}{$\bar{c}$}             & \multicolumn{1}{c|}{$1/2$} & \multicolumn{1}{c|}{$2/3$}   & $1280 \pm 30$                                  \\ \cline{2-7} 
    \multicolumn{1}{|c|}{}                            & \multicolumn{1}{c|}{strange}           & \multicolumn{1}{c|}{s}               & \multicolumn{1}{c|}{$\bar{s}$}             & \multicolumn{1}{c|}{$1/2$} & \multicolumn{1}{c|}{$-1/3$}  & $96_{-4}^{+8}$                                 \\ \hline
    \multicolumn{1}{|c|}{\multirow{2}{*}{\textbf{3}}} & \multicolumn{1}{c|}{top}               & \multicolumn{1}{c|}{t}               & \multicolumn{1}{c|}{$\bar{t}$}             & \multicolumn{1}{c|}{$1/2$} & \multicolumn{1}{c|}{$2/3$}   & $172100 \pm 600$                               \\ \cline{2-7} 
    \multicolumn{1}{|c|}{}                            & \multicolumn{1}{c|}{bottom}            & \multicolumn{1}{c|}{b}               & \multicolumn{1}{c|}{$\bar{b}$}             & \multicolumn{1}{c|}{$1/2$} & \multicolumn{1}{c|}{$-1/3$}  & $4180_{-30}^{+40}$                             \\ \hline
    \multicolumn{7}{|c|}{\textbf{Leptons}}                                                                                                                                                                                                                                                                               \\ \hline
    \multicolumn{1}{|c|}{\multirow{2}{*}{\textbf{1}}}                & \multicolumn{1}{c|}{electron}          & \multicolumn{1}{c|}{$e^-$}           & \multicolumn{1}{c|}{$\bar{e}^-$}           & \multicolumn{1}{c|}{$1/2$} & \multicolumn{1}{c|}{-1}              & $0.511$                                        \\ \cline{2-7} 
    \multicolumn{1}{|c|}{}                   & \multicolumn{1}{c|}{electron neutrino} & \multicolumn{1}{c|}{$\nu_{e}$}       & \multicolumn{1}{c|}{$\bar{\nu}_e$}         & \multicolumn{1}{c|}{$1/2$} & \multicolumn{1}{c|}{0}               & $<0.0000022$                                   \\ \hline
    \multicolumn{1}{|c|}{\multirow{2}{*}{\textbf{2}}} & \multicolumn{1}{c|}{muon}              & \multicolumn{1}{c|}{$\mu^-$}         & \multicolumn{1}{c|}{$\bar{\mu}^-$}         & \multicolumn{1}{c|}{$1/2$} & \multicolumn{1}{c|}{-1}              & $105.7$                                        \\ \cline{2-7} 
    \multicolumn{1}{|c|}{}                            & \multicolumn{1}{c|}{muon neutrino}     & \multicolumn{1}{c|}{$\nu_{\mu}$}     & \multicolumn{1}{c|}{$\bar{\nu}_{\mu}$}     & \multicolumn{1}{c|}{$1/2$} & \multicolumn{1}{c|}{0}               & $<0.170$                                       \\ \hline
    \multicolumn{1}{|c|}{\multirow{2}{*}{\textbf{3}}} & \multicolumn{1}{c|}{tau}               & \multicolumn{1}{c|}{$\tau$}          & \multicolumn{1}{c|}{$\bar{\tau}$}          & \multicolumn{1}{c|}{$1/2$} & \multicolumn{1}{c|}{-1}              & $1776.86 \pm 0.12$                             \\ \cline{2-7} 
    \multicolumn{1}{|c|}{}                            & \multicolumn{1}{c|}{tau neutrino}      & \multicolumn{1}{c|}{$\nu_{\tau}$}    & \multicolumn{1}{c|}{$\bar{\tau}_{\tau}$}   & \multicolumn{1}{c|}{$1/2$} & \multicolumn{1}{c|}{0}               & $< 15.5$                                       \\ \hline
    \end{tabular}
    \caption{Table showing properties of all the fermions, including name, symbol, antiparticle, spin, charge, generation and mass. The antiparticle for d and b also have a bar over, but it is on top of the box surrounding the letter. }
    \label{tab:fermion_table}
    \end{table}
Another mystery regarding the neutrinos is the relation to antiparticles. To each particle corresponds an antiparticle, i.e. for $l^-\to l^+$, $q\to \bar{q}$, where the antiparticle 
and particle are different. With neutrinos however it is not known whether the neutrinos and antineutrinos are the same particle $\nu = \bar{\nu}$ (Majorana neutrino), 
or if they are different $\nu \neq \bar{\nu}$ (Dirac neutrino).
\par
Quarks are fractional charge particles, with defined charge of either $2/3$ or $-1/3$, as shown in table \ref{tab:fermion_table}. They are the main building blocks of protons and neutrons, and are bound by the strong 
force, the strongest of the four fundamental forces. The force mediator is the gluon. The other half of fermions are the leptons. They are split into the charged 
leptons (electrons, muons and taus), and the uncharged leptons (neutrinos). The charged leptons can interact via the electroweak force, where the Z, W bosons 
as well as the photon can be a mediator. Note that the neutrinos can only interact through the weak force.

\subsection*{Bosons}
Bosons are integer number spin particles, with spin $0, 1, 2, ...$. Within bosons there are so-called elementary bosons, some 
of which are force carriers or mediators such as the $W^{\pm}$, Z and the photon. The Higgs boson is also an elementary scalar 
boson, but it is not a force carrier. It is the excitation of the Higgs field which is reponsible for providing masses for the 
fermions via a process called spontaneous symmetry breaking\cite{Pich:819632}. Other bosons are so-called composite bosons, mesons 
($q\bar{q}$) and baryons ($qqq$) which are particles constructed by even or half integer spin. Bosons also have antiparticles, 
where $(\gamma, g, H^{0}, Z^{0})$ are equal to their respective antiparticle, and the $W^{-} \to W^{+}$ are not equal.

\subsection*{Feynman diagrams}
A graphical way to understand particle interactions are through Feynman diagrams. Feynman diagrams 
are drawn based on the Feynman rules for a given Lagrangian\cite{Pich:819632}\cite{feynrules}, 
and each component can be linked to a part in the Lagrangian for the system. 

\begin{figure}[H]
    \centering
    
\begin{tikzpicture}
    \begin{feynman}
    
        \vertex at (0.5, 1.) (a2){\(e^{-}\)} ;
        \vertex at (0.5, -1.)  (a5){\(e^{+}\)} ;

        \vertex at (1.5, 0.0) (c);
        \vertex at (3., 0.0) (d);

        \vertex at (4, 1.) (f1) {\(\mu^{-}\)} ;
        \vertex at (4,-1.) (f2) {\(\mu^{+}\)};

        
    \diagram*{


        (a2) -- [fermion] (c) -- [boson] (d),
    
        (a5) -- [anti fermion] (c) -- [boson, edge label = {$\gamma, Z$}] (d),
        (d) -- [fermion] (f1),
        (d) -- [anti fermion] (f2),
        
        ;
    };
    \end{feynman}
    \end{tikzpicture}
    \caption[Electron scattering diagram]{Feynman diagram of muon pair production from electron scattering. Here, both Z and $\gamma$ can work as the propagator.}
    \label{fig:eemm_scat}
\end{figure}

In figure \ref{fig:eemm_scat} we have a Feynman diagram describing electron-positron annihilation into muon-antimuon pairs. In this thesis, all 
diagrams will be interpreted from left to right, i.e. \hl{figure} \ref{fig:eemm_scat}.
The diagram contains all the components in the Lagrangian, and arrows, curly lines and so on all have its own meaning. A straight line with 
an arrow usually means a fermion, where the direction of the arrow tells if the particle is a particle(arrow towards the vertex) or an antiparticle (arrow away from the vertex). 
There is often also a propagator between the left and right side of the Feynman diagram, 
and they depend on the processes we want to study. In the diagram above we have lepton scattering, thus we can both have the photon and the 
Z-boson as a propagator. This process is called a neutral current\cite{Pich:819632}, as the total charge coming out of the interaction is 0. 
As with neutral currents, we also have so-called charged currents, where the sum of charge is not 0. Note that we only require charge conservation,
thus there is nothing wrong with either having a neutral or a charged current, as long as charge conservation is preserved. \par 
Feynman diagrams are used for both visualizing scatterings and decays. An example is provided in figure \ref{fig:mw_decay}. 

\begin{figure}[H]
    \centering
    
\begin{tikzpicture}
    \begin{feynman}
    
        \vertex at (0.5, 0.) (a2){\(\mu^{-}\)} ;

        \vertex at (2., 0.0) (c);
        \vertex at (3., -1.) (d);
        
        \vertex at (3.5, 0.7) (a3){\(\nu_{\mu}\)} ;

        \vertex at (4, -0.5) (f1){\(e^{-}\)} ;
        \vertex at (4.,-1.4) (f2) {\(\bar{\nu}_{e}\)};

        
    \diagram*{


        (a2) -- [fermion] (c) -- [boson, edge label = {$W^{-}$}] (d),
        (a2) -- [fermion] (c) -- [fermion] (a3),
        (d) -- [fermion] (f1),
        (d) -- [anti fermion] (f2),
        
        ;
    };
    \end{feynman}
    \end{tikzpicture}
    \caption[Muon decay diagram]{Muon decay into an electron, an electron neutrino and a muon neutrino via the $W^{-}$ boson. Read the graph from left to right.}
    \label{fig:mw_decay}
\end{figure}

In figure \ref{fig:mw_decay} we have a decay of a muon into and electron and two neutrinos through a charged current. 
\marginpar{Lang setning} The examples above in figure \ref{fig:eemm_scat} and \ref{fig:mw_decay} show interactions with 
the electroweak force, but along with the electroweak interactions, are also the quantum chromodynamics (QCD), 
responsible for interactions between quarks and gluons. A strange property of QCD,
is that the coupling constant $\alpha_{S}$, unlike the $\alpha_{EM}$ for electromagnetism,
gets stronger as the energy decreases. This is because QCD (and weak interactions) are based on non-abelian 
groups\cite{Peskin:1995ev}, thus to study such interactions, one needs to create collisions at very high energies. 

\subsubsection*{Proton-proton collisions}
Proton-proton collisions require high energy to occur. During an interaction between the protons both the quarks and gluons
can interact as shown below:

\begin{figure}[H]
    \centering
    
\begin{tikzpicture}
    \begin{feynman}
    
        \vertex at (0.5, 1.) (a2){\(q\)} ;
        \vertex at (0.5, -1.)  (a5){\(\bar{q}\)} ;

        \vertex at (1.5, 0.0) (c);
        \vertex at (3., 0.0) (d);

        \vertex at (4, 1.) (f1) {\(l^{-}\)} ;
        \vertex at (4,-1.) (f2) {\(l^{+}\)};

        
    \diagram*{


        (a2) -- [fermion] (c) -- [boson] (d),
    
        (a5) -- [anti fermion] (c) -- [boson, edge label = {$\gamma, Z$}] (d),
        (d) -- [fermion] (f1),
        (d) -- [anti fermion] (f2),
        
        ;
    };
    \end{feynman}
    \end{tikzpicture}
    \caption[$q\bar{q}$ collision into lepton pair]{Proton-proton collision with lepton pair production via the Z boson or photon. Read from left to right.}
    
    \label{fig:qqbar_ll}
\end{figure}

\begin{figure}[H]
    \centering
    
\begin{tikzpicture}
    \begin{feynman}
    
        \vertex at (0, 1.) (a2){\(g\)} ;
        \vertex at (0, -1.)  (a5){\(g\)} ;

        \vertex at (1.5, 1.) (c);
        \vertex at (3., 1.) (d);

        \vertex at (1.5, -1.) (e);
        \vertex at (3., -1.) (f);

        \vertex at (1.5, 1.) (g);
        \vertex at (1.5, -1.) (h);

        \vertex at (3, 1) (a1);
        \vertex at (4.5, 1.5) (a3);

        \vertex at (3, 1) (a6);
        \vertex at (4.5, 0.5) (a7);


        \vertex at (3, -1) (b1);
        \vertex at (4.5, -1.5) (b2);

        \vertex at (3, -1) (b3);
        \vertex at (4.5, -0.5) (b4);

       
        
    \diagram*{


        (a2) -- [gluon] (c) -- [anti fermion, edge label = {$\bar{t}$}] (d),
    
        (a5) -- [gluon] (e) -- [fermion, edge label = {$t$}] (f),

        (g) -- [fermion, edge label = {$t$}] (h),

        (a1) -- [boson, edge label ={$w^-$}] (a3),

        (a6) -- [anti fermion, edge label ={$\bar{b}$}] (a7),

        (b1) -- [boson, edge label ={$w^+$}] (b2),

        (b3) -- [anti fermion, edge label ={$b$}] (b4),
        
        ;
    };
    \end{feynman}
    \end{tikzpicture}
    \caption[gluon-gluon interaction into $t\bar{t}$ production]{Proton-proton collision showing the $t\bar{t}$ channel. Read from left to right.}

    \label{fig:ttbar_feynman_example}
    \end{figure}

In figure \ref{fig:qqbar_ll} and figure \ref{fig:ttbar_feynman_example} we have two examples of Feynman
diagrams for possible interactions in proton-proton collisions. The first figure displays a lepton pair production
via the Z boson or photon from quark-antiquark annihilation, and the second figure displays the $t\bar{t}$ production
via gluon-gluon fusion. \par

\subsection*{Some limitations}
Although the SM has had great success comparing predictions with experiments,
there are still several problems not addressed by it. Gravity is one example. The SM
as described above cannot incorporate gravity in a quantized way. There 
are models that try, without success so far, to address this problem. They supplement the SM,
and do not derrive from it. Another problem with the SM is a curious property of the weak interaction, 
namely that parity is broken. Parity as a mathematical operation is equivalent to the spatial inversion 
through the origin\cite{Thomson:2013zua}:
\begin{equation}
    x \to -x.
\end{equation}
In other words, parity can be thought of as left-right symmetry, or mirror symmetry. Breaking of parity is observed
in weak currents, where the mediator of the charged currents $W^{\pm}$, only interacts with 
left-handed fermions and right-handed antifermions. In the SM, neutrinos are assumed to be massless, and the righthanded 
neutrinos are sterile, i.e. they do not interact in the SM. \par 
This asymmetry is strange, and hints towards new physics that perhaps can restore the parity breaking at much higher energies. 
Another note to make is that it has been experimentally verified that the neutrinos are massive\cite{Katrin_neutrinos},
with an upper limit on the mass for the anti electron neutrino of $m_{\nu} < 0.8eVc^2$ at $90\%$ confidence level.
This is direct experimental evidence that the SM is wrong, as the tiny masses of the neutrinos are not predicted by the SM. 








