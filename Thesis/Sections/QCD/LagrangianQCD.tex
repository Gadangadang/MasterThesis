In this chapter we take a closer look at QCD as the qunatum field theory of the strong interaction. We begin by introducing the Lagrangian and its constituents leading up to a discussion on the notion of asymptotic freedom and the running of the strong coupling. Then the process of DIS is explored, where important concepts as factorization and parton distribution functions are introduced. An operator valued expression for the parton distributions is derived and we show how Wilson lines can be used to render these gauge invariant. We also introduce perturbative parton-in-parton distributions by dressing them with Wilson lines. Thereafter, we investigate the Drell-Yan process and calculate a NLO quark-antiquark annihilation process. We show that even after renormalization and regularization of this cross section that there are certain regions of phase space where it gain large logarithmic corrections.

\section{Field Theoretical Description}
In \cref{chap:Geometry of gauge theories} we introduced a geometric formulation behind non-Abelian gauge theories. We are now ready to build on this formalism and introduce the non-Abelian gauge theory of the strong interaction, namely Quantum Chromodynamics. It is formulated in terms of quark and gluon fields, where quarks are spin 1/2-fermions and gluons are spin-1 gauge bosons. The symmetry group is $SU(3)_{c}$, meaning that the fields carry a quantum number which we call colour. 

The quark fields live in the fundamental representation of the gauge group, and is represented as a triplet in colour space
\begin{align}
    \psi(x)= \begin{pmatrix}
           \psi_{1}(x) \\
           \psi_{2}(x) \\
           \psi_{3}(x)
         \end{pmatrix}\,,
\end{align}
which we know transform under local gauge transformations as
\begin{align}
    \psi(x)\rightarrow U(x)\psi(x)\,,
\end{align}
where
\begin{align}
    U(x)=e^{i\alpha^{a}(x)t^{a}}\,.
\end{align}
The generators $t^{a}$ are $3\times 3$ hermitian matrices, where the hermiticity follows by insisting that the group matrices $U$ are unitary, while keeping the group parameters $\alpha^{a}(x)$ real. The generators of $SU(3)$ are traceless, and the group index runs over eight values $a=1,\dots,3^{2}-1=8$. This has the implication that there are three different coloured quarks and there are eight gluons. However, experimentally we know that there are three families of quarks, which are represented as doublets
\begin{align}
    \begin{pmatrix}
           u \\
           d \\
         \end{pmatrix}\,,
         \begin{pmatrix}
           c \\
           s \\
         \end{pmatrix}\,,
         \begin{pmatrix}
           t \\
           b \\
         \end{pmatrix}\,,
\end{align}
giving that there are in total 36 different quarks. In a scattering calculations we will not explicitly specify whether we are dealing with an up-quark or a down-quark, just regard them as fermions with fractional electric charge and remember to sum over all of them. We will mostly work in the high energy limit where we use that the quarks are massless, and then we tend to ignore the heaviest generation of the top and bottom quarks. 

From \cref{eq:Lie commutator} we know that the generators satisfy the commutation relation
\begin{align}
    [t^{a},t^{b}]=if^{abc}t^{c}\,,
\end{align}
in which the structure constants are real and antisymmetric in the indices $a,b$ and $c$. The normalization of the structure constants fixes the trace and Casimir invariants for any representation $R$, i.e.
\begin{align}
    \text{tr}(t^{a}t^{b})=C(R)\delta^{ab}\,,\hspace{1cm}\sum_{a}t^{a}t^{a}=C_{2}(R)\,.
\end{align}
For the fundamental representation of the quarks we have that
\begin{align}
    C(\text{fund})&=T_{F}=\frac{1}{2}
    \\
    C_{2}(\text{fund})&=C_{F}=\frac{4}{3}\,,
\end{align}
and for the adjoint representation of the gluons we have that
\begin{align}
    C(\text{adj})&=C_{2}(\text{adj})=C_{A}=3\,.
\end{align}

The classical QCD Lagrangian follows directly from the Yang-Mills Lagrangian \cref{eq:Yang-Mills Lagrangian}, and the quantized from the Fadeev-Popov Lagrangian \cref{eq:Fadeev-Popov Lagrangian}. Using axial gauge we have that the gauge fixed QCD-Lagrangian is given by
\begin{align}
     \mathcal{L}_{QCD}=-\frac{1}{4}(F_{\mu\nu}^{a})^{2}+\Bar{\psi}(i\slashed{D}-m)\psi-\frac{1}{2\xi}(n^{\mu}A_{\mu}^{a})^{2}\,.
\end{align}
where it is understood that the quark fields carry both flavour index and colour index, i.e. $\psi=(\psi_{\alpha\,i})$ where $\alpha=u,d,s,c,b,t$ and $i=1,2,3$. Expanding this Lagrangian will give all Feynman rules needed for calculating Feynman diagrams. The expansion is straightforward using \cref{eq:field strenght in wilson chapter}, but tedious. We can group the different parts in the following way\footnote{These are the bare fields, but since there are too many subscripts and superscripts already we will not write this out explicitly.}
\begin{align}\label{eq:QCD splitted LAgrangian}
    \mathcal{L}_{QCD}=\mathcal{L}_{\text{GF}}+\mathcal{L}_{\text{Dirac}}+\mathcal{L}_{\text{int}}\,,
\end{align}
where
\begin{align}
    \mathcal{L}_{\text{GF}}&=-\frac{1}{4}\big(\partial_{\mu}A_{\nu}^{a}-\partial_{\nu}A_{\mu}^{a}\big)^{2}-\frac{1}{2\xi}(n^{\mu}A_{\mu}^{a})^{2}\,,
    \\
    \mathcal{L}_{\text{Dirac}}&=\Bar{\psi}(i\slashed{\partial}-m_{0})\psi\,,
    \\
    \mathcal{L}_{\text{int}}=&g_{0}\Bar{\psi}\slashed{A}^{a}t^{a}\psi-g_{0}f^{abc}(\partial_{\mu}A_{\nu}^{a})A^{\mu\,b}A^{\nu\,c}\nonumber
    \\
    &-\frac{1}{2}g_{0}^{2}f^{abe}f^{ecd}A_{\mu}^{a}A_{\nu}^{b}A^{\mu\,c}A^{\nu\,d}\,.\label{eq:QCD interaction Lagrangian}
\end{align}
The first term of \cref{eq:QCD interaction Lagrangian} describes the interaction between quarks and gluons, the second term the interaction between three gluons and the fourth term between four gluons. These terms give rise to the following vertex rules\footnote{In the three gluon vertex all momenta are pointing towards the vertex.}

\begin{fmffile}{qqg}
\begin{align}
\begin{gathered}
\begin{fmfgraph*}(80,50)
\fmfleft{i1,i2}
\fmfright{o1}
\fmfv{label=$a\hspace{0.1cm}\mu$}{o1}
\fmf{fermion,tension=1/3}{i1,v1}
\fmf{plain}{v1,v2}
\fmf{fermion}{v2,v3}
\fmf{plain}{v3,i2}
\fmf{gluon}{v1,o1}
\end{fmfgraph*}
\end{gathered}\hspace{1cm}&=ig\gamma^{\mu}t^{a}\,,\label{eq:quark-gluon vertex}
\\\nonumber\\
\begin{gathered}
\begin{fmfgraph*}(80,50)
\fmfleft{i1,i2}
\fmfright{o1}
\fmfv{label=$a\hspace{0.1cm}\mu$}{i2}
\fmfv{label=$b\hspace{0.1cm}\nu$}{i1}
\fmfv{label=$c\hspace{0.1cm}\rho$}{o1}
\fmf{gluon,label=$p$, l.side=left,tension=1/3}{i1,v1}
\fmf{gluon,label=$k$, l.side=left,tension=1/3}{i2,v1}
\fmf{gluon,label=$q$, l.side=left}{v1,o1}
\end{fmfgraph*}
\end{gathered}\hspace{1cm}&=gf^{abc}\big(g^{\mu\nu}(k-p)^{\rho}+g^{\nu\rho}(p-q)^{\mu}+g^{\rho\mu}(q-k)^{\nu}\big)\,,\label{eq:three gluon vertex}
\\\nonumber\\\nonumber\\
\begin{gathered}
\begin{fmfgraph*}(80,50)
\fmfleft{i1,i2}
\fmfright{o1,o2}
\fmfv{label=$a\hspace{0.1cm}\mu$}{i2}
\fmfv{label=$c\hspace{0.1cm}\rho$}{i1}
\fmfv{label=$d\hspace{0.1cm}\sigma$}{o1}
\fmfv{label=$b\hspace{0.1cm}\nu$}{o2}
\fmf{gluon,tension=1/3}{i1,v1}
\fmf{gluon,tension=1/3}{i2,v1}
\fmf{gluon,tension=1/3}{v1,o1}
\fmf{gluon,tension=1/3}{v1,o2}
\end{fmfgraph*}
\end{gathered}\hspace{1cm}&=-ig^{2}\big[f^{abc}f^{cde}(g^{\mu\rho}g^{\nu\sigma}-g^{\mu\sigma}g^{\nu\rho})+f^{ace}f^{bde}(g^{\mu\nu}g^{\rho\sigma}-g^{\mu\sigma}g^{\nu\rho})\nonumber\\&\hspace{1cm}+f^{ade}f^{bce}(g^{\mu\nu}g^{\rho\sigma}-g^{\mu\rho}g^{\nu\sigma})\big]\,.\label{eq:four gluon vertex}\\\nonumber
\end{align}
\end{fmffile}
The propagators for fermions and gauge bosons can be found in e.g. \cref{eq:canocical fermion propagator} and \cref{eq:covariant propagator}.

The missing piece to this Lagrangian is to renormalize it by rescaling the fields and define counterterms that remove all UV-divergences. We have already seen in \cref{sec:renormalized perturbation theory} how this works for a scalar theory. We will use that the exact same procedure applies for QCD, with the obvious difference that there are several fields and a more intricate relation between the parameters and the renormalization constants. We rescale the bare fields in the usual way 
\begin{align}
    A^{\mu}&\rightarrow\mathcal{Z}_{3}^{1/2}A^{\mu}\,,\label{eq:rescaled gauge field QCD}
    \\
    \psi&\rightarrow\mathcal{Z}_{2}^{1/2}\psi\,,
\end{align}
where we need different renormalization constants for different fields. We have neglected all subscript separating bare and renormalized fields, but keep in mind that with this rescaling it is the renormalized fields that appear in the \textquote{new} Lagrangian. We can now define the counterterms
\begin{align}\label{eq:counterterms}
    \delta_{1}&=Z_{1}-1\,,\hspace{1cm}\delta_{2}=\mathcal{Z}_{2}-1\,,\hspace{1cm}\delta_{3}=\mathcal{Z}_{3}-1\,,\hspace{1cm}
    \\\vspace{0.2cm}
    \delta_{m}&=\mathcal{Z}_{2}m_{0}-m\,,\hspace{1cm}\delta_{A^{3}}=\mathcal{Z}_{A^{3}}-1\,,\hspace{1cm}\delta_{A^{4}}=\mathcal{Z}_{A^{4}}-1\,,
\end{align}
where we defined
\begin{align}\label{eq:relation g and g0}
    \mathcal{Z}_{1}g&=\mathcal{Z}_{2}\mathcal{Z}_{3}^{1/2}g_0\,,\hspace{1cm}\mathcal{Z}_{A^{3}}g=\mathcal{Z}_{3}^{3/2}g_{0}\,,\hspace{1cm}\mathcal{Z}_{A^{4}}g^{2}=\mathcal{Z}_{3}^{2}g_{0}^{2}\,,
\end{align}
such that the Lagrangian takes the form
\begin{align}
    \mathcal{L}=\mathcal{L}_{\text{R}}+\mathcal{L}_{\text{CT}}\,,
\end{align}
where the first part is identical to \cref{eq:QCD splitted LAgrangian}, but with the renormalized fields and parameters. The counterterm part is given by
\begin{align}\label{eq:QCD counterterm Lagrangian}
    \mathcal{L}_{\text{CT}}=&-\frac{1}{4}\delta_{3}\big(\partial_{\mu}A_{\nu}^{a}-\partial_{\nu}A_{\mu}^{a}\big)^{2}-\frac{1}{2\xi}\delta_{3}(n^{\mu}A_{\mu}^{a})^{2}+\delta_{2}\Bar{\psi}(i\slashed{\partial}-\delta_{m})\psi\,,
    \\
    &+g\delta_{1}\Bar{\psi}\slashed{A}^{a}t^{a}\psi-g\delta_{A^{3}}f^{abc}(\partial_{\mu}A_{\nu}^{a})A^{\mu\,b}A^{\nu\,c}\nonumber
    \\
    &-\frac{1}{2}g^{2}\delta_{A^{4}}f^{abe}f^{ecd}A_{\mu}^{a}A_{\nu}^{b}A^{\mu\,c}A^{\nu\,d}\,.
\end{align}

The definition of the counterterms only makes sense if one gives a precise definition for the physical mass and coupling, i.e. one has to define renormalization conditions to put constraints on the counterterms. For example, the physical mass of the quarks are defined as the pole of the quark propagator at all orders, just like we did for $\phi^{4}$-theory. 

The notion of running coupling in non-Abelian gauge theories is different than in Abelian gauge theories or scalar theories. The concept of changing with scale is not different, but how it changes with scale. In \cref{eq:solution of beta function} we found that the coupling in $\phi^{4}$-theory increases with the change of scale. This is also true for an Abelian gauge theory like QED, but for a non-Abelian gauge theory like QCD this is no longer true. In QCD the coupling is large in the low energy regime, and perturbation theory breaks down, while it becomes smaller at higher energies. This phenomena is known as \emph{asymptotic freedom} and is only present in non-Abelian gauge theories. To see how this comes about, we have to look at the beta function of the theory. The beta function in QCD is found by calculating the quark self-energy (giving $\mathcal{Z}_2$), corrections to the gluon propagator (giving $\mathcal{Z}_3$) and corrections to the vertex function (giving $\mathcal{Z}_1$). This is a very long and tedious calculation, so we will not perform it explicitly here, but instead state the results in order to understand the behaviour of the strong coupling.

In \cref{eq:relation g and g0} the relation between the bare coupling and renormalized coupling is given by
\begin{align}
    g_0&=\frac{\mathcal{Z}_{1}}{\mathcal{Z}_{2}\mathcal{Z}_{3}^{1/2}}g\mu^{\epsilon}\,,
\end{align}
where we have inserted $\mu^{\epsilon}$ to make the coupling dimensionless in dimensional regularization. We can now use that the bare coupling is independent on $\mu$, giving
\begin{align}
    0=\mu\dv{g_0}{\mu}=\mu\dv{}{\mu}\Big[\frac{\mathcal{Z}_{1}}{\mathcal{Z}_{2}\mathcal{Z}_{3}^{1/2}}g\mu^{\epsilon}\Big]\,,
\end{align}
and performing the differentiation will give the differential equation
\begin{align}
    \mu\dv{g}{\mu}=-\epsilon g-g\Big[\frac{\mu}{\mathcal{Z}_1}\dv{\mathcal{Z}_1}{\mu}-\frac{\mu}{\mathcal{Z}_2}\dv{\mathcal{Z}_2}{\mu}-\frac{1}{2}\frac{\mu}{\mathcal{Z}_3}\dv{\mathcal{Z}_3}{\mu}\Big]\,.
\end{align}
The $\mu$ dependence on $\mathcal{Z}_{i}$ is only through the coupling $g$, so we can write
\begin{align}
    \frac{\mu}{\mathcal{Z}_i}\dv{\mathcal{Z}_i}{\mu}=\frac{1}{\mathcal{Z}_{i}}\pdv{\mathcal{Z}_{i}}{\mu}\mu\pdv{g}{\mu}=\frac{1}{\mathcal{Z}_{i}}\pdv{\mathcal{Z}_{i}}{\mu}\beta\,,
\end{align}
and use that $\mathcal{Z}_{i}=1+\delta_{i}$, where the relevant counterterms can be found in \cite{Schwartz:2013pla}\footnote{They use a different convention in dimensional regularization, with $d=4-\epsilon$. The difference is just a factor of two, so we adjust the answer to suit our convention.},
\begin{align}
    \delta_{1}&=-\frac{1}{\epsilon}\frac{g^{2}}{(4\pi)^{2}}(C_{F}+C_{A})
    \\
    \delta_{2}&=-\frac{1}{\epsilon}\frac{g^{2}}{(4\pi)^{2}}C_{F}
    \\
    \delta_{3}&=\frac{1}{\epsilon}\frac{g^{2}}{(4\pi)^{2}}\big(\frac{5}{3}C_{A}-\frac{4}{3}n_{f}T_{F}\big)\,,
\end{align}
where $n_f$ is the number of flavours. We observe that the counterterms are defined at $\mathcal{O}(g^{2})$. We are only interested in the one-loop correction and since all one-loop diagrams have $\mathcal{O}(g^{3})$, we can write
\begin{align}
    \frac{1}{\mathcal{Z}_{i}}\pdv{\mathcal{Z}_{i}}{\mu}=\pdv{\delta_{i}}{g}+\cdots\,,
\end{align}
and use that
\begin{align}
    \beta=-\epsilon g+\mathcal{O}(g^{2})\,,
\end{align}
leading to the following differential equation
\begin{align}\label{eq:beta one-loop}
    \mu\dv{g}{\mu}=-\epsilon g+\epsilon g^{2}\pdv{}{g}\big(\delta_{1}-\delta_{2}-\frac{1}{2}\delta_{3}\big)=-\epsilon g-\frac{g^{3}}{(4\pi)^{2}}\big(\frac{11}{3}C_{A}-\frac{4}{3}n_{f}T_{F}\big)\,.
\end{align}
We can safely send $\epsilon\rightarrow 0$, and define
\begin{align}\label{eq:zeroth order beta function}
    \beta_{0}=\frac{11}{3}C_{A}-\frac{4}{3}n_{f}T_{F}\,.
\end{align}

The solution to \cref{eq:beta one-loop} can be found by using separation of variables, giving
\begin{align}\label{eq: g running coupling one-loop}
    g^{2}(\mu)=\frac{g^{2}(\mu_0)}{1+\frac{g^{2}(\mu_{0})}{(4\pi)^{2}}\beta_{0}\ln(\frac{\mu^{2}}{\mu_{0}^{2}})}\,.
\end{align}
In scattering observables we will mostly use that $\alpha_s=g^{2}/4\pi$, so we can write \cref{eq: g running coupling one-loop} as
\begin{align}\label{eq:one-loop strong coupling}
    \alpha_{s}(\mu)=\frac{\alpha_{s}(\mu_0)}{1+\frac{\alpha_{s}(\mu_0)}{4\pi}\beta_0\ln(\frac{\mu^{2}}{\mu_{0}^{2}})}\,,
\end{align}
which is the well known one-loop coupling of QCD. The crucial point here is that: as long as $n_f<17$, $\beta_0>0$, and the coupling decreases as a function of increasing $\mu$. We know that there are six flavours of quarks, so QCD is an asymptotically free theory. This discovery was made independetly by D.Pulitzer, F.Wilczek and D.Gross, for which they received the 2004 Nobel Prize in physics.

The notion of asymptotic freedom is very counter-intuitive, as it implies that particles at an infinitesimally small separation do not attract each other (colourwise). There is a metaphor for this kind of behaviour, which is also useful to explain what we call \emph{colour confinement}. Imagine we have two particles connected by a rubber band. If we move the particles closer and closer each other, the rubber band will relax and nothing happens. The moment we try to move the particles apart, the tension in the band will increase, and the more we pull trying to separate the particles, the stronger the tension. Hence, it can work as a metaphor for why coloured particles has never been seen to exist as free states, i.e. they are confined to exist as colour neutral composite particles. We can actually extend the analogy even further. Imagine that we stretch the rubber band with such force that it snaps, i.e. we have succeeded in separating them. However, the amount of energy needed to snap the gluon binding energy is so large that a particle-antiparticle pair is created and binds with the particles we try to separate, giving new colour neutral states. However, this is only an analogy as confinement is still not theoretically understood.

The closest attempt to understand and explain confinement is due to Kenneth Wilson by the use of Wilson loops on the lattice. It is well known that the attempt to define a expectation value of a QCD current describing the potential between two colour charges fails. This procedure works in QED, and can be shown to be equivalent to calculating the beta function, i.e. it gives a measure of the running of the coupling. The problem with this approach in QCD is that it is not gauge invariant. This led Kenneth Wilson to postulate that this potential could be described by instead defining the expectation value of a Wilson loop on the lattice. Wilson's idea was that if the expectation value of a Wilson loop could be shown to be proportional to the area of the Wilson loop, it would indicate that the potential between colour charges grew with distance. Wilson was able to show analytically that on the lattice the expectation value of a Wilson loop scales as the area of the loop, which indicates that confinement is present in QCD \cite{Wilson:74}. The problem is that Wilson's arguments is valid for any gauge theory, but QED does most certainly not have this behaviour. There is also a problem with preserving confinement in the continuum limit, i.e. when the lattice spacing is removed. However, there has been a lot of progress in this field the last decades, and that is due to the development of generalized loops and loop calculus, see e.g. \cite{TAVARES_1994,MIGDAL1983199,KORCHEMSKY1986459}. We will not pursue confinement and the low energy regime any further, but it is worth mentioning that Wilson loops/lines are at the centre of both perturbative and non-perturbative QCD. 

Apart from small coupling at large energies, the other consequence of asymptotic freedom is that at a given energy scale, the coupling becomes large and invalidates perturbation theory. In theories without asymptotic freedom, like QED, this scale is so large that it is not relevant to consider, but for QCD we have that for $\Lambda_{\text{QCD}}\approx 250 $MeV the coupling becomes larger than one. The consequence of this is that we can not use perturbation theory to understand the low energy behaviour of bound states. This separates QCD into two regimes; a part where $\mu>\Lambda_{\text{QCD}}$ where perturbation theory is valid and a region where we can not calculate. This latter part needs to be extracted from experiment, and this is the basis for the \emph{factorization} framework we will investigate in more detail in this chapter. 