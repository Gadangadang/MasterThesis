\section*{Testing strategy}

Before testing the AE and VAE on signal samples, it is of interest to test the sensitivity of both the regular and variational autoencoders 
on alterings on the Monte Carlo. This can be thought of as initial testing. 
\subsection*{Channel removing}
As the goal of the autoencoder is to reconstruction data is has looked on, one idea was to remove on of the channels in the standard model.
The idea was that some of the channels differs enough in the final states they produce and thus the RMMs for the events in the given selection. 
All channels where tested on as signal, but one can expect some to have more similar results than others. 

\subsection*{Altering transverse momentum}
Another idea for anomaly detection testing with the MonteCarlo was to alter the transverse momentum of some of the particles. Random events where
selected and had the transverse energy changed, in accordance with equation \ref{eq:et}. The hope is that especially events with above 5 time 
increase in transverse momentum should be picked up. Note that by changing the transverse momentum, the change in transverse energy also changes. 
From equation \ref{eq:deltaet} we have that a scale change k in $p_T$ yields the following new relation:

\begin{equation}\label{eq:deltaet_scale}
    \delta\boldsymbol{e_T}^k = \frac{kE_T(i_n-1) - E_T(i_n)}{kE_T(i_n-1) + E_T(i_n)}, \, n = 2, ..., N.
\end{equation}
Thus, both the transverse energy and the change in transverse energy is changed for this test.


\subsection*{Signal testing}
It would not technically be true to call the machine learning used in this thesis as unsupervised learning. 
This is mostly due to the fact that we show it labeled MonteCarlo to train on, and thus essentially gives 
the models a target to aim for. But, sinse we do not show it any signals in training, it can be allowed to 
name the learning strategy as semi-supervised learning. Because of this, one does not bias oneself too much 
if one then tests the algorithm on one or more signal samples. This type of testing can be separated into two
different categories, the open signal testing and the blind signal testing. \par 
The open signal testing first takes a trained model that has only seen SM MC and then runs inference on both 
SM MC and the given signal. Then the search strategy mentioned in the previous section is done on those 
collected reconstruction error distributions. The blind signal testing is the final test, where a dataset 
is prepared incapsulating SM MC, and one or multiple signals in a mix. The labels are kept, for verification. 
That way, if one were to do open signal testing on many signal samples from many different theories, one could 
in principle bias oneself too much. This way one can get an unbiased measurement of the performance. 