\section*{Testing strategy}

Before testing the AE and VAE on signal samples, it is interesting to test the sensitivity of 
both the regular and variational autoencoders on anomalous SM MC events. This can be thought of as initial testing. 
\subsection*{Channel removing}
One idea was to test the anomaly detection capability by removing one of the channels in the SM.
The idea was that some of the channels differ enough in the final states and thus in their RMM signature that
they might be treated like anomalous data by the autoencoders. All channels were tested on as signal.
\subsection*{Altering transverse momentum}
Another idea for anomaly detection testing with the SM MC was to alter the transverse momentum of some of the particles. Random events were
selected, and their transverse energy was changed, in accordance with equation \ref{eq:et}. The hope is that especially events with above 5 time 
increase in transverse momentum should be picked up. Note that by changing the transverse momentum, the change in transverse energy also changes. 
From equation \ref{eq:deltaet} we have that a scale change k in $p_T$ yields the following new relation:

\begin{equation}\label{eq:deltaet_scale}
    \delta\boldsymbol{e_T}^k = \frac{kE_T(i_n-1) - E_T(i_n)}{kE_T(i_n-1) + E_T(i_n)}, \, n = 2, ..., N.
\end{equation}
Thus, both the transverse energy and the change in transverse energy is changed for this test.


\subsection*{Signal testing}
It is not correct to call the machine learning algorithm used in this thesis as unsupervised learning. 
This is mostly due to the fact that we show it labeled MonteCarlo to train on, and thus essentially gives 
the models a target to aim for. But, since we do not show the autoencoders any signals in training, it can be allowed to 
name the learning strategy as semi-supervised learning. Because of this, one does not bias oneself too much 
if the algorithm is tested on one or more signal samples. This type of testing can be separated into two
different categories, the open signal testing and the blind signal testing. \par 
The open signal testing first takes a trained model that has only seen SM MC and then runs inference on both 
SM MC and the given signal. Then the search strategy mentioned in the previous section is done on those 
collected reconstruction error distributions. The blind signal testing is the final test, where a dataset 
is prepared incapsulating SM MC or ATLAS data, and one or multiple signals. The labels are kept, for verification. 
That way, if one were to do open signal testing on many signal samples from many theories, one could 
in principle bias oneself too much. This way one can get an unbiased measurement of the performance. 