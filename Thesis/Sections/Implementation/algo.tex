\section{Code implementation}
\subsection*{Machine learning implementation}

The machine learning analysis was written with Keras\cite{chollet2015keras} using the Tensorflow api\cite{tensorflow-whitepaper}. 
The machine learning structure was written using a functional structure\footnote{Functional structure uses a function call for layers, i.e. for layers a, b, then b(a) will connect the two layers, and equals a sequential link a $\to$ b. This allows for more flexible structures. More on the functional api can be found \href{https://www.tensorflow.org/guide/keras/functional}{here}.}.
In practice, this model could just as well have been written as a Sequential model\footnote{Sequential structure adds layers in sequence, i.e. for layers a, b, c we have that a $\to$ b $\to$ c, with a strict structure. This allows for more organized code. More on sequential models can be found \href{https://www.tensorflow.org/guide/keras/sequential_model}{here}.}, 
but at a cost of flexibility and lack of the possibility for creating non-linear structure in the architecture. The code consists of one general class for
 the autoencoder, where the different testing cases are different classes inheriting from the parent class.
\subsection*{Construction of a neural network in Tensorflow}

Using the functional structure, a general neural network in the Tensorflow API can be constructed as shown in code example \ref{code:python_func_example_general}.

\begin{figure}[h!]
    \centering
\begin{lstlisting}[language=Python, style=pythonstyle, label={code:python_func_example_general}]
import tensorflow as tf


inputs = tf.keras.layers.Input(shape=data_shape, name="input")

# First hidden layer
First_layer = tf.keras.layers.Dense(
    units=30,
    activation="relu"
)(inputs)

# Second hidden layer
Second_layer = tf.keras.layers.Dense(
    units=45, 
    activation="relu"
)(First_layer)

# Second hidden layer
output_layer = tf.keras.layers.Dense(
    units=1, 
    activation="sigmoid"
)(Second_layer)


# Model definition
nn_model = tf.keras.Model(inputs, output_layer, name="nn_model")

hp_learning_rate = 0.0015
optimizer = tf.keras.optimizers.Adam(hp_learning_rate)
nn_model.compile(loss="mse", optimizer=optimizer, metrics=["mse"]) 
\end{lstlisting}
\caption[Functional structure]{Functional structure for Tensorflow neural network.}
\label{code:functional}
\end{figure}
The neural network in the code above contains one input layer, one hidden layer and an output layer. The choice of nodes and activation functions are 
arbitrary here as the use case has not been defined. Note that code example \ref{code:sequatial} is exactly the same as code example \ref{code:functional},
but using the sequential structure.

\begin{figure}[h!]
    \centering
\begin{lstlisting}[language=Python, style=pythonstyle, label={code:python_seq_example}]
import tensorflow as tf

nn_model = tf.keras.Sequential(
    [
        tf.keras.layers.Dense(30, activation="relu", input_shape=data_shape),
        tf.keras.layers.Dense(45, activation="relu"),
        tf.keras.layers.Dense(1, activation="sigmoid"),
    ]
)

hp_learning_rate = 0.0015
optimizer = tf.keras.optimizers.Adam(hp_learning_rate)
nn_model.compile(loss="mse", optimizer=optimizer, metrics=["mse"]) 
\end{lstlisting}
\caption[Sequential structure]{Sequential structure for Tensorflow neural network.}
\label{code:sequatial}
\end{figure}

\subsection*{Data handling python side}


%Optimizing \cite{ADAM:opti}, \par 
%Plotting \cite{Hunter:2007}, \par
%More plotting  \cite{Waskom2021}, \par

\subsubsection*{Implementation of the RMM matrix}
Examples of RMM matrices used in this thesis are shown in figure \ref{fig:rmm_singular_events}. 


\begin{figure}[h!]
    \centering
    \begin{subfigure}{.8\textwidth}
        \includegraphics[width=\textwidth]{Figures/rmms/rmm_event_6993776_diboson4L.pdf}
        \caption{RMM matrix for event number 6993776 from the Monte Carlo diboson4L event. Each feature is scaled based on a fit for that feature for 
        all events in the training set ($\approx 80\%$ of total MC). This event contains two ljets, one electron and two muons.}
        \label{fig:rmm_dib4l_event}
    \end{subfigure}
    \hfill
    \begin{subfigure}{.8\textwidth}
        \includegraphics[width=\textwidth]{Figures/rmms/rmm_event_11739638_higgs.pdf}
        \caption{ RMM matrix for event number 11739638 from the Monte Carlo Higgs event. Each feature is scaled based on a fit for that feature for 
        all events in the training set ($\approx 80\%$ of total MC). This event contains five ljets, one bjet, one electron and two muons. }
        \label{fig:rmm_higgs_event}
    \end{subfigure}
    \hfill        
    \caption[Single event RMM plot]{Two RMM matrices for one diboson4L (\ref{fig:rmm_dib4l_event}) event and one Higgs (\ref{fig:rmm_higgs_event}) event.}
    \label{fig:rmm_singular_events}
\end{figure}

The two RMM matrices in figure \ref{fig:rmm_singular_events} are created from two different channels in the MC samples. 
This RMM is of type T4N5\footnote{T4 $\to$ 4 particle types: bjets, ljets, electrons and muons. N5 $\to$ 5 particles per 
particle type. Note here that we have 5 particles only for the leptons, and 6 particles for each of the types of jets, so it is a almost T4N5 matrix.}. 
 For easier interpretability, the gray area corresponds to a missing value, leading to so-called "islands" in the RMM matrix.
 Note here that the y-axis for the RMM's lack every other label, due to lack of space in the y-axis of the plot. If looked more 
 closely upon, one can see that each figure has all RMM cells, just that the labels, which are identical to the x-axis label, 
 only show for 1, 3, 5, ... The RMM plots were created using Plotly\cite{plotly}.


\subsubsection*{Setup for 3 lepton dataset}
The 3 lepton dataset is about 96 GB of data or cirka 381873 events when implementing the RMM structure for 6 b- and ljets, 5 electrons and 5 muons. The dataset 
is converted from a ROOT N-tuple using RDataframe to a Pandas dataframe\cite{reback2020pandas} for further preprocessing. Having added the channel name column\footnote{This column would in a fully supervised setting be used as a target vector, but in this thesis it will only be used for legends in
histograms and to index out certain channels in the validation and training set. },  weights, 
trilepton mass\footnote{Invariant mass of three leptons. This is assured to exist from event selection in the 3 lepton dataset.} as well as the RMM structure, 
the datasets are divided into 
a training and a validation/test set in an 80-20 split\footnote{80 percent are resrved for training and 20 percent are reserved for testing/validation. 
This convention is standard within the machine learning community.}. The split was done using the $train\_test\_split()$ function from the Scikit-learn library\cite{scikit-learn}.
Then the data and SM MC was scaled using the MinMax scaler via the "$.fit\_transform()$" and "$.transform()$" functions from the 
Scikit-learn library. The training and validation/test set are then converted to numpy arrays\cite{harris2020array} for faster 
loading and easier indexing, and saved as ".npy" files. This allows for faster reuse of the arrays. 

\subsubsection*{Setup via iterative training for 2 lepton dataset}
The two lepton dataset contains about 1.5 - 2 TB of data or about 119291900 events when implementing the RMM structure for 6 
b- and ljets, 5 electrons and 5 muons, as for the 3 lepton dataset. This is too much to hold in memory at the same time, thus it had to 
be split into several smaller datasets, called megasets. Figure \ref{fig:2lep_struct} visualize the structure used. 

\begin{figure}[h!]
    \centering
    \includegraphics[width=0.6\linewidth]{Figures/2lep_config/megaset_struct.jpeg}
    \caption[Megaset structure diagram]{Megaset structure for the 2lep dataset. This figure generalises to M channels, and N megasets, 
    where M and N does not have to be equal. The more you increase the number of megasets, 
    the smaller each megaset will be in bytesize, but in order to keep the natural distribution in the SM MC distribution, it is not recommended to make too small sets. 
    In this thesis 10 megasets were created where the simulated data contained 7 channels. }
    \label{fig:2lep_struct}
\end{figure}

Pandas is not built for very large datasets since it is not running in a parallalized way. To handle this, the library
Polars\footnote{Polars uses all available cores on the system and has excellent memory 
handling capability, see \href{https://pola-rs.github.io/polars-book/user-guide/}{Polars User Guide}}
\cite{ritchie_vink_2023_7744139} was used instead. When all channels were split, a merging was done combining all the channels in a given 
megaset to a separate dataset. The selection of events from each channel was done randomly, 
which is important, as we want to the best of our ability keep the distribution signature of the entire dataset in each megaset. If not, the model will 
be biased towards those datasets with the most events. Once each of the megasets where merged, the training could begin in an iterative fashion. Because
Tensorflow is statically compiled, one cannot call the fit function over and over again. Instead, the weights trained based on one megaset is stored and 
reloaded into a new model, thus the weights are still trained on the entire set, but in a batch like manner. 
The implementation used in this thesis is shown in code example \ref{code:training}.



\begin{figure}[h!]
    \centering
\begin{lstlisting}[language=Python, style=pythonstyle, label={code:megabatch_training}]
for megaset in range(totmegasets):
    
    #* Load model 

    autoencoder = getModel()
    
    if megaset != 0:
        autoencoder.load_weights('./checkpoints/Megabatch_checkpoint')
        
        
    #* Run Training
    with tf.device("/GPU:0"):

        tf.config.optimizer.set_jit("autoclustering")

        autoencoder.fit(
            xtrain,
            xtrain,
            epochs=epochs,
            batch_size=b_size,
            validation_data=(xval, xval),
            sample_weight=x_train_weights,
        )
        
    
    AE_model.save_weights('./checkpoints/Megabatch_checkpoint')

\end{lstlisting}
\caption[Tensorflow neural network training]{Example code for training a neural network on megasets with Tensorflow.}
\label{code:training}
\end{figure}

