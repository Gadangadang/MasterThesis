\section{Code implementation}

The machine learning analysis was written with Keras\cite{chollet2015keras} using the Tensorflow api\cite{tensorflow2015-whitepaper} . 
The machine learning structure was written using a functional structure\footnote{Functional structure uses a function call for layers, i.e for layers a,b, then b(a) will connect the two layers, and equals a sequential link a $\to$ b. This allows for more flexible structures. More on the functional api can be found \href{https://www.tensorflow.org/guide/keras/functional}{here}.}
In practise, this model could just as well have been written as a Sequential model\footnote{Sequential structure adds layers in sequence, i.e for layers a, b, c we have that a $\to$ b $\to$ c, with a strict structure. This allows for more organized code. More on sequential models can be found \href{https://www.tensorflow.org/guide/keras/sequential_model}{here}.}, 
but at a cost of flexibility and lack of potential non-linear structure in the architecture. The code consists of one general class for
 the autoencoder, where the different testing cases are different classes inheriting from the parent class.\par


%Tuning \cite{omalley2019kerastuner}, \par 
%Tuning optimization \cite{hyperband:opt}, \par
%Optimizing \cite{ADAM:opti}, \par 
%Splitting and scaling \cite{scikit-learn}, \par
%Plotting \cite{Hunter:2007}, \par
%More plotting  \cite{Waskom2021}, \par
%Even more plotting \cite{plotly}, \par
%Arrays \cite{harris2020array}, \par 
%Dataframe structure \cite{reback2020pandas}, \par


\subsection*{Construction of a neural network in Tensorflow}

Using the functional structure, a general neural network in the Tensorflow API can be constructed as shown below. 
\begin{lstlisting}[language=Python, style=pythonstyle, label={code:python_func_example_general}]
import tensorflow as tf


inputs = tf.keras.layers.Input(shape=data_shape, name="input")

# First hidden layer
First_layer = tf.keras.layers.Dense(
    units=30,
    activation="relu"
)(inputs)

# Second hidden layer
Second_layer = tf.keras.layers.Dense(
    units=45, 
    activation="relu"
)(First_layer)

# Second hidden layer
output_layer = tf.keras.layers.Dense(
    units=1, 
    activation="sigmoid"
)(Second_layer)


# Model definition
nn_model = tf.keras.Model(inputs, output_layer, name="nn_model")

hp_learning_rate = 0.0015
optimizer = tf.keras.optimizers.Adam(hp_learning_rate)
nn_model.compile(loss="mse", optimizer=optimizer, metrics=["mse"]) 
\end{lstlisting}
The neural network here contains one input layer, two hidden layers, and an output layer. The choice of nodes and activation functions are 
arbitrary here as the use case has not been defined. Note that this is exactly the same as the previous example, but using the sequential structure.


\begin{lstlisting}[language=Python, style=pythonstyle, label={code:python_seq_example}]
import tensorflow as tf

nn_model = tf.keras.Sequential(
    [
        tf.keras.layers.Dense(30, activation="relu", input_shape=data_shape),
        tf.keras.layers.Dense(45, activation="relu"),
        tf.keras.layers.Dense(1, activation="sigmoid"),
    ]
)

hp_learning_rate = 0.0015
optimizer = tf.keras.optimizers.Adam(hp_learning_rate)
nn_model.compile(loss="mse", optimizer=optimizer, metrics=["mse"]) 
\end{lstlisting}

In the case of this project, we want to create an autoencoder, as shown in figure \ref{fig:ae_denoise}. A general network archtitecture is proposed below, 
where the nodes, activation functions, and other hyperparameters can be tuned. 


\begin{lstlisting}[language=Python, style=pythonstyle, label={code:ae_code_arch}]
import tensorflow as tf

class HyperParameterTuning(RunAE):
    def __init__(self, data_structure: object, path: str)->None:
        super().__init__(data_structure, path)

    def AE_model_builder(self, hp: kt.engine.hyperparameters.HyperParameters):

        alpha_choice = hp.Choice("alpha", values=[1.0, 0.5, 0.1, 0.05, 0.01])

        # Activation functions
        activations = {
            "relu": tf.nn.relu,
            "tanh": tf.nn.tanh,
            "leakyrelu": "leaky_relu",
            "linear": tf.keras.activations.linear,
        }  # lambda x: tf.nn.leaky_relu(x, alpha=alpha_choice),

        # Input layer
        inputs = tf.keras.layers.Input(shape=self.data_shape, name="encoder_input")

        # First hidden layer
        x = tf.keras.layers.Dense(
            units=hp.Int(
                "num_of_neurons1", min_value=60, max_value=self.data_shape - 1, step=1
            ),
            activation=activations.get(
                hp.Choice("1_act", ["relu", "tanh", "leakyrelu", "linear"])
            ),
        )(inputs)

        val = hp.Int("lat_num", min_value=1, max_value=9, step=1)

        # Forth hidden layer
        x2 = tf.keras.layers.Dense(
            units=val,
            activation=activations.get(
                hp.Choice("4_act", ["relu", "tanh", "leakyrelu", "linear"])
            ),
        )(x)

        # Encoder definition
        encoder = tf.keras.Model(inputs, x2, name="encoder")

        # Latent space
        latent_input = tf.keras.layers.Input(shape=val, name="decoder_input")

        # Output layer
        output = tf.keras.layers.Dense(
            self.data_shape,
            activation=activations.get(
                hp.Choice("8_act", ["relu", "tanh", "leakyrelu", "linear"])
            ),
        )(latent_input)

        # Encoder definition
        decoder = tf.keras.Model(latent_input, output, name="decoder")

        # Output definition
        outputs = decoder(encoder(inputs))

        # Model definition
        AE_model = tf.keras.Model(inputs, outputs, name="AE_model")

        hp_learning_rate = hp.Choice(
            "learning_rate", values=[9e-2, 9.5e-2, 1e-3, 1.5e-3]
        )
        optimizer = tf.keras.optimizers.Adam(hp_learning_rate)

        AE_model.compile(loss="mse", optimizer=optimizer, metrics=["mse"])

        return AE_model
    \end{lstlisting}

This function creates a tensorflow.keras model that has tuneable structure, thus allowing for optimized tuning with the Keras-Tuner 
library\cite{omalley2019kerastuner}. The model is then compiled and saved with the selected architecture from the tuning. 
Note that the tuning section does not train the model, it simply creates a set of copies to check and stores the hyperparameters. 
Then, the model is trained with the selected hyperparameters and inference is calculated. 