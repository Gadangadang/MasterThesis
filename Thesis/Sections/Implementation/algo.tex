\section*{Code implementation}

The machine learning analysis was written with Keras using the Tensorflow api\cite{tensorflow2015-whitepaper} \cite{chollet2015keras}. 
The machine learning structure was written using a functional structure\footnote{Functional structure uses a function call for layers, i.e for layers a,b, then b(a) will connect the two layers, and equals a sequential link a $\to$ b. This allows for more flexible structures. More on the functional api can be found \href{https://www.tensorflow.org/guide/keras/functional}{here}.}
In practise, this model could just as well have been written as a Sequential model\footnote{Sequential structure adds layers in sequence, i.e for layers a, b, c we have that a $\to$ b $\to$ c, with a strict structure. This allows for more organized code. More on sequential models can be found \href{https://www.tensorflow.org/guide/keras/sequential_model}{here}.}, 
but at a cost of flexibility and lack of potential non-linear structure in the architecture.\par

%Tuning \cite{omalley2019kerastuner}, \par 
%Tuning optimization \cite{hyperband:opt}, \par
%Optimizing \cite{ADAM:opti}, \par 
%Splitting and scaling \cite{scikit-learn}, \par
%Plotting \cite{Hunter:2007}, \par
%More plotting  \cite{Waskom2021}, \par
%Even more plotting \cite{plotly}, \par
%Arrays \cite{harris2020array}, \par 
%Dataframe structure \cite{reback2020pandas}, \par