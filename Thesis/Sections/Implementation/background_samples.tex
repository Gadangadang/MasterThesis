\section*{Background samples}

\subsection*{3 lepton background MonteCarlo}
To look for the heavy neutrinos, we need to train on background MonteCarlo with that finalstate aswell. This means in a sense that 
we want the autoencoder to learn what is expected from the Standard model in terms of this final state. The 3 lepton background MonteCarlo 
contains the following channels:

\begin{enumerate}
    \item Wjets
    \item Triboson
    \item Higgs
    \item Zttjets
    \item Zmmjets
    \item Zeejets
    \item SingleTop
    \item TopOther
    \item ttbar
    \item Diboson2L
    \item Diboson3L
    \item Diboson4L
\end{enumerate}

Below are three examples of what some of the samples contains. The selected ones are likely Feynman diagrams for $t\bar{t}$, Higgs and Zeejets channels. 

\begin{figure}[h!]
    \centering
    \caption{Proton-proton collision showing the $t\bar{t}$ channel. Here the w bosons decay leptonically and one or more jets
    are misreconstructed as leptons by the detector. }

\begin{tikzpicture}
    \begin{feynman}
    
        \vertex at (0, 1.) (a2){\(g\)} ;
        \vertex at (0, -1.)  (a5){\(g\)} ;

        \vertex at (1.5, 1.) (c);
        \vertex at (3., 1.) (d);

        \vertex at (1.5, -1.) (e);
        \vertex at (3., -1.) (f);

        \vertex at (1.5, 1.) (g);
        \vertex at (1.5, -1.) (h);

        \vertex at (3, 1) (a1);
        \vertex at (4.5, 1.5) (a3);

        \vertex at (3, 1) (a6);
        \vertex at (4.5, 0.5) (a7);


        \vertex at (3, -1) (b1);
        \vertex at (4.5, -1.5) (b2);

        \vertex at (3, -1) (b3);
        \vertex at (4.5, -0.5) (b4);

       
        
    \diagram*{


        (a2) -- [gluon] (c) -- [anti fermion, edge label = {$\bar{t}$}] (d),
    
        (a5) -- [gluon] (e) -- [fermion, edge label = {$t$}] (f),

        (g) -- [fermion, edge label = {$t$}] (h),

        (a1) -- [boson, edge label ={$w^-$}] (a3),

        (a6) -- [anti fermion, edge label ={$\bar{b}$}] (a7),

        (b1) -- [boson, edge label ={$w^+$}] (b2),

        (b3) -- [anti fermion, edge label ={$b$}] (b4),
        
        ;
    };
    \end{feynman}
    \end{tikzpicture}
    \label{fig:ttbar_feynman}
    \end{figure}

    \begin{figure}[h!]
        \centering
        \caption{Proton-proton collision showing the Higgs channel. Here the Z bosons decay leptonically. }
    
    \begin{tikzpicture}
        \begin{feynman}
        
            \vertex at (0, 1.) (a2){\(g\)} ;
            \vertex at (0, -1.)  (a5){\(g\)} ;
    
            \vertex at (1.5, 1.) (c);
            \vertex at (3., 0) (d);
    
            \vertex at (1.5, -1.) (e);
            \vertex at (3., 0) (f);
    
            \vertex at (1.5, 1.) (g);
            \vertex at (1.5, -1.) (h);
    
            \vertex at (3, 0) (a1);
            \vertex at (4.5, 0) (a3);

            \vertex at (6, 1) (a6);
            \vertex at (6, -1) (a7);
    
           
            
        \diagram*{
    
    
            (a2) -- [gluon] (c) -- [anti fermion, edge label = {$\bar{t}$}] (d),
        
            (a5) -- [gluon] (e) -- [fermion, edge label = {$t$}] (f),
    
            (g) -- [fermion] (h),
    
            (a1) -- [dashed, edge label = {$H$}] (a3),

            (a3) -- [boson, edge label = {$Z$}] (a6),
            (a3) -- [boson, edge label = {$Z$}] (a7),


            
            ;
        };
        \end{feynman}
        \end{tikzpicture}
        \label{fig:higgs_feynman}
        \end{figure}
        
        \begin{figure}[h!]
            \centering
            \caption{Proton-proton collision showing the Zeejets channel. Here one of the Z bosons decay leptonically and the W boson
            decays hadronically. }
        
        \begin{tikzpicture}
            \begin{feynman}
                \vertex at (0, 1.) (a2){\(q\)} ;
                \vertex at (0, -1.)  (a5){\(\bar{q}^{'}\)} ;

                \vertex at (1.5, 1.) (c);

                \vertex at (1.5, -1.) (e);

                \vertex at (3, 1.) (b1);

                \vertex at (3, -1.) (b2);

                \vertex at (4.5, -0.5) (b4);

                \vertex at (4.5, -1.5) (b3);

                \vertex at (4.5, 0.5) (b5);

                \vertex at (4.5, 1.5) (b6);
        
                
               
                
            \diagram*{
        
        
                (a2) -- [fermion] (c),
            
                (a5) -- [anti fermion] (e),

                (c) -- [fermion] (e),

                (c) -- [boson, edge label={$Z$}] (b1),
                (e) -- [gluon, edge label={$g$}] (b2),

                (b2) -- [fermion, edge label={$b$}] (b4),
                (b2) -- [anti fermion, edge label={$\bar{b}$}] (b3),

                (b1) -- [fermion, edge label={$e^-$}] (b5),
                (b1) -- [anti fermion, edge label={$e^+$}] (b6),
        
                ;
            };
               
            \end{feynman}
            \end{tikzpicture}
            \label{fig:zeejets_feynman}
            \end{figure}

\subsection*{2 lepton background MonteCarlo}

\begin{enumerate}
    \item singletop 
    \item Diboson 
    \item Zeejets 
    \item Zmmjets 
    \item Zttjets 
    \item Wjets 
    \item ttbar
   
\end{enumerate}

 