\section{Background samples}

To look for anomalies in the 2 or 3 lepton final states we need to train on background Monte Carlo samples with that specific final state. 
This means in a sense that we want the autoencoder to learn what is expected from the SM in terms of this final state. 
The 2 and 3 lepton + $e_T^{miss}$ background Monte Carlo contains the following channels:

\begin{table}[h!]
    \centering
    
    \begin{tabular}{|cc|}
    \hline
    \multicolumn{1}{|c|}{\textbf{3 lepton + $e_T^{miss}$}} & \textbf{2 lepton + $e_T^{miss}$} \\ \hline
    \multicolumn{2}{|c|}{\textbf{Channel names:}}                                             \\ \hline
    \multicolumn{1}{|c|}{Wjets}& Wjets                            \\ \hline
    \multicolumn{1}{|c|}{ttbar}& ttbar                            \\ \hline
    \multicolumn{1}{|c|}{Singletop}& Singletop                        \\ \hline
    \multicolumn{1}{|c|}{ZeeJets}  & ZeeJets                          \\ \hline
    \multicolumn{1}{|c|}{ZmmJets}                          & ZmmJets                          \\ \hline
    \multicolumn{1}{|c|}{ZttJets}                          & ZttJets                          \\ \hline
    \multicolumn{1}{|c|}{Higgs}                            & Diboson                          \\ \hline
    \multicolumn{1}{|c|}{Triboson}                         &                                  \\ \hline
    \multicolumn{1}{|c|}{TopOther}                         &                                  \\ \hline
    \multicolumn{1}{|c|}{Diboson2L}                        &                                  \\ \hline
    \multicolumn{1}{|c|}{Diboson3L}                        &                                  \\ \hline
    \multicolumn{1}{|c|}{Diboson4L}                        &                                  \\ \hline
\end{tabular}
\caption[SM MC channels]{SM MC channels for both the 2 and 3 lepton + $e_T^{miss}$ final state background. }
\label{tab:bkg_channels}
\end{table}


Below are three Feynamn diagrams, two of which are both represented in the 2 and 3 lepton + $e_T^{miss}$ background MonteCarlo shown in table \ref{tab:bkg_channels}. 
The selected ones are likely Feynman diagrams for $t\bar{t}$, Higgs and Zeejets channels. The $t\bar{t}$ process shown in figure \ref{fig:ttbar_feynman}
leads to two leptons + $e_T^{miss}$ from the W-decays, as well as to two b-jets. One jet can be mis-identified as a lepton. 

\begin{figure}[h!]
    \centering
    

\begin{tikzpicture}
    \begin{feynman}
    
        \vertex at (0, 1.) (a2){\(g\)} ;
        \vertex at (0, -1.)  (a5){\(g\)} ;

        \vertex at (1.5, 1.) (c);
        \vertex at (3., 1.) (d);

        \vertex at (1.5, -1.) (e);
        \vertex at (3., -1.) (f);

        \vertex at (1.5, 1.) (g);
        \vertex at (1.5, -1.) (h);

        \vertex at (3, 1) (a1);
        \vertex at (4.5, 1.5) (a3);

        \vertex at (3, 1) (a6);
        \vertex at (4.5, 0.5) (a7);


        \vertex at (3, -1) (b1);
        \vertex at (4.5, -1.5) (b2);

        \vertex at (3, -1) (b3);
        \vertex at (4.5, -0.5) (b4);

       
        
    \diagram*{


        (a2) -- [gluon] (c) -- [anti fermion, edge label = {$\bar{t}$}] (d),
    
        (a5) -- [gluon] (e) -- [fermion, edge label = {$t$}] (f),

        (g) -- [fermion, edge label = {$t$}] (h),

        (a1) -- [boson, edge label ={$w^-$}] (a3),

        (a6) -- [anti fermion, edge label ={$\bar{b}$}] (a7),

        (b1) -- [boson, edge label ={$w^+$}] (b2),

        (b3) -- [fermion, edge label ={$b$}] (b4),
        
        ;
    };
    \end{feynman}
    \end{tikzpicture}
    \caption[$t\bar{t}$ production diagram]{Proton-proton collision showing the $t\bar{t}$ channel. Here the w bosons decay leptonically and one or more jets
    can be misreconstructed as fake leptons by the detector. }
    \label{fig:ttbar_feynman}
    \end{figure}

    The Higgs process in figure \ref{fig:higgs_feynman} can lead to four leptons. One or two leptons can be misidentified and therefor constitute a background 
    for the 2 lepton + $e_T^{miss}$ or 3 lepton + $e_T^{miss}$ analyses.
    \begin{figure}[h!]
        \centering
        
    \begin{tikzpicture}
        \begin{feynman}
        
            \vertex at (0, 1.) (a2){\(g\)} ;
            \vertex at (0, -1.)  (a5){\(g\)} ;
    
            \vertex at (1.5, 1.) (c);
            \vertex at (3., 0) (d);
    
            \vertex at (1.5, -1.) (e);
            \vertex at (3., 0) (f);
    
            \vertex at (1.5, 1.) (g);
            \vertex at (1.5, -1.) (h);
    
            \vertex at (3, 0) (a1);
            \vertex at (4.5, 0) (a3);

            \vertex at (6, 1) (a6);
            \vertex at (6, -1) (a7);
    
           
            
        \diagram*{
    
    
            (a2) -- [gluon] (c) -- [anti fermion, edge label = {$\bar{t}$}] (d),
        
            (a5) -- [gluon] (e) -- [fermion, edge label = {$t$}] (f),
    
            (g) -- [fermion] (h),
    
            (a1) -- [dashed, edge label = {$H$}] (a3),

            (a3) -- [boson, edge label = {$Z$}] (a6),
            (a3) -- [boson, edge label = {$Z$}] (a7),


            
            ;
        };
        \end{feynman}
        \end{tikzpicture}
        \caption[Higgs production diagram]{Proton-proton collision showing the Higgs channel. Here the Z bosons decay leptonically, leading to a four lepton final state. }
    
        \label{fig:higgs_feynman}
        \end{figure}
        
        The Z + jets process in figure \ref{fig:zeejets_feynman} can lead to two leptons and two b-jets. If one or both of the jets are fakd as leptons,
        the process constitutes a source of background for the 3 lepton + $e_T^{miss}$ or 2 lepton + $e_T^{miss}$ analysis.
        \begin{figure}[h!]
            \centering
            
        
        \begin{tikzpicture}
            \begin{feynman}
                \vertex at (0, 1.) (a2){\(q\)} ;
                \vertex at (0, -1.)  (a5){\(\bar{q}^{'}\)} ;

                \vertex at (1.5, 1.) (c);

                \vertex at (1.5, -1.) (e);

                \vertex at (3, 1.) (b1);

                \vertex at (3, -1.) (b2);

                \vertex at (4.5, -0.5) (b4);

                \vertex at (4.5, -1.5) (b3);

                \vertex at (4.5, 0.5) (b5);

                \vertex at (4.5, 1.5) (b6);
        
                
               
                
            \diagram*{
        
        
                (a2) -- [fermion] (c),
            
                (a5) -- [anti fermion] (e),

                (c) -- [fermion] (e),

                (c) -- [boson, edge label={$Z$}] (b1),
                (e) -- [gluon, edge label={$g$}] (b2),

                (b2) -- [fermion, edge label={$b$}] (b4),
                (b2) -- [anti fermion, edge label={$\bar{b}$}] (b3),

                (b1) -- [fermion, edge label={$e^-$}] (b5),
                (b1) -- [anti fermion, edge label={$e^+$}] (b6),
        
                ;
            };
               
            \end{feynman}
            \end{tikzpicture}
            \caption[Zeejets channel diagram]{Proton-proton collision showing the Zeejets channel. Here one of the Z bosons decay leptonically and the gluon
            decays hadronically. }
            \label{fig:zeejets_feynman}
            \end{figure}

\newpage