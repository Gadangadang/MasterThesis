\section{The search strategy}
The strategy used to look for anomalies in the three lepton + missing energy finals state is presented as follows. 
First, the MonteCarlo and ATLAS data 
for training and inference are constructed with the RMM from section \ref{sec:rmm} as features. After scaling and 
splitting, $80\%$ of the MC will be used for 
training the neural network, and the remaining $20\%$ will be used for inference. To separate the anomalies, the 
reconstruction error of the input data 
will create a distribution. The same will then be done for a test signal, and eventually the ATLAS data. The 
ATLAS data is completely unlabeled which 
makes validation difficult. However, the test signals are labeled, thus we can analyze the reconstruction error 
distribution of those signals as a 
weak\footnote{Weak means here that allthough the test signals can provide valuable insight in how the autoencoder 
can separate anomalies,
 we must be cautious about generalizing the evaluation value of those results. There are many contributing factors 
 that affect the reconstruction error separation, 
 which will be such as similiarities with the Standard model, large amounts of missing energy, high energy particles 
 and more,  discussed in chapter \ref{Chap:results_discuss}. } 
 validation of performance when later running on ATLAS data. 
 
Standard analyses creates what is called a signal region. The signal region is a region in the feature space where 
the signal is maximized. We use this region to calculate the 
significance of a result in the search, which is really the only metric that is of use. The statistical uncertainty 
and noise is proportional to the amount of SM MC, thus
with lower amounts of SM MC, the better the significance will be. Using the reconstruction error, the autoencoder 
can create its own signal region, 
namely the areas of high reconstruction error. A cut here will then be used to separate the anomalies from the 
Standard model in for example missing transverse energy, or 
other features of interest. \par 

The signal region for the regular autoencoder models were created by 
calculating the median $m_{err}$ of the reconstruction error. Then, 3 cuts were made, starting at 
$m_{err} + im_{err}/5$ for $i = 1,2,3$. This was a direct result of the shape of the background reconstruction error, being a hill like 
shape. The median then became a good place to start to remove alot of the background. This is however just a guess 
for an optimal signal region, as the true signal is unknown, and the method has to be as unbiased as possible. There is however 
an issue to keep in mind here. The method to find the cut is in some sense based 
on the slope shape of the SM MC reconstruction error distribution, thus three cuts based on the median seems 
like a good choice. However, if one 
then uses all the event in the signal region, it might be that one misses the ideal amount of background and 
signal to create the significance. Thus, 
for each reconstruction error cut, there is an associated graph showing the significance as a function of 
$e_T^{miss}$. More specifically, the function 
calculates from a point and outwards, for all values in the $e_T^{miss}$ distribution. Thus, you can find the ideal cut, 
within the signal region, for where to choose the amount of background and signal to get a better significance.\par 
To avoid bias by the author, 
some of the test signals should contain some signal samples that the creator of the model has not seen before, 
to ensure no changes have been made 
 to the network to adjust for that signal. ROC curves will then be used to evaluate the binary classification 
 ability of the autoencoder. 
 \par 
