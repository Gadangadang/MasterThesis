\section{The search strategy}
The strategy used to look for anomalies in the three lepton + missing energy finals state is presented as follows. First, the MonteCarlo and ATLAS data 
for training and inference are constructed with the RMM from section \ref{sec:rmm} as features. After scaling and splitting, $80\%$ of the MC will be used for 
training the neural network, and the remaining $20\%$ will be used for inference. To separate the anomalies, the reconstruction error of the input data 
will create a distribution. The same will then be done for a test signal, and eventually the ATLAS data. The ATLAS data is completely unlabeled which 
makes validation difficult. However, the test signals are labeled, thus we can analyze the reconstruction error distribution of those signals as a 
weak\footnote{Weak means here that allthough the test signals can provide valuable insight in how the autoencoder can separate anomalies,
 we must be cautious about generalizing the evaluation value of those results. There are many contributing factors that affect the reconstruction error separation, 
 which will be such as similiarities with the Standard model, large amounts of missing energy, high energy particles and more,  discussed in chapter \ref{Chap:results_discuss}. } 
 validation of performance when later running on ATLAS data. 
 
Standard analyses creates what is called a signal region. The signal region is a region in the feature space where the signal is maximized. We use this region to calculate the 
significance of a result in the search, which is really the only metric that is of use. The statistical uncertainty and noise is proportional to the amount of SM MC, thus
with lower amounts of SM MC, the better the significance will be. Using the reconstruction error, the autoencoder can create its own signal region, 
namely the areas of high reconstruction error. A cut here will then be used to separate the anomalies from the Standard model in for example missing transverse energy, or 
other features of interest. 


To avoid bias by the author, 
some of the test signals should contain some signal samples that the creator of the model has not seen before, to ensure no changes have been made 
 to the network to adjust for that signal. ROC curves will then be used to evaluate the binary classification ability of the autoencoder. 