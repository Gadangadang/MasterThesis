\section{The ATLAS detector}
The ATLAS detector comprises several main components for recording proton-proton collisions. Amongst them are the muon detectors, 
the electromagnetic calorimeters, the hadronic calorimeters and the inner detector. The kinematics and geometry describing the collisions are 
written below.
\subsection*{Kinematics and detector geometry}
Before the data can be analysed, it has to be collected and processed in a detector. The data used in the work of this thesis 
are generated from proton-proton collisions in the ATLAS detector at the Large Hadron Collider (LHC) during the Run 2 period from 2015 to 2018. 
The ATLAS inner detector itself is contained in a solenoid field, 
and the kinematic variables are measured based on the following coordinate system. The z-axis is defined to go 
along the center axis of the solenoid, whereas the y-axis points upwards in the detector and the x-axis radialy 
outwards from the center axis. This allows for all transverse variables to be defined in the x-y plane\cite{Gramstad:1631043}. 
From this we contruct the azimuthal and polar angles $\phi$ and $\theta$, where the azimuthal angle $\phi$ is the angle around the z-axis, 
and the polar angle $\theta$ is the angle from the z-axis, as shown in figure \ref{fig:long_trans_plane}.


\begin{figure}[h!]
    \centering
    \begin{subfigure}{.45\textwidth}
        \includegraphics[width=\textwidth]{Figures/atlas/transverse_plane.png}
        \caption{Transverse plane of detector}
        \label{fig:transverse_plane}
    \end{subfigure}
    \hfill
    \begin{subfigure}{.45\textwidth}
        \includegraphics[width=\textwidth]{Figures/atlas/longitudal_plane.png}
        \caption{Longitudal plane of detector}
        \label{fig:longitudal_plane}
    \end{subfigure}
    \hfill
    \caption[ATLAS detector longitudal and azimuthal diagrams]{Spherical coordinate definitions with the azimuthal and polar angles $\phi$ and $\theta$. Here figure \ref{fig:transverse_plane} of the transverse plane shows the z-axis
     into the paper, where as figure \ref{fig:longitudal_plane} of the longitudal plane shows the positive x-axis going out of the paper. Figure taken from \cite{Gramstad:1631043}.}
    \label{fig:long_trans_plane}
\end{figure}

In figure \ref{fig:long_trans_plane} the kinematic variables for proton-proton collisions are described by the energy, rest mass 
and momentum, given as E, m, and $\textbf{p} = (p_x, p_y, p_z)$ respectively. As the particles move with very high energy, we will 
use the relativistic four momentum\footnote{In special relativity, four momentum is used as both energy and momentum are conserved, and 
thus by creating the four momentum, you achieve a Lorentz invariant quantity\cite{Thomson:2013zua}. }, given as $\textbf{P} = (E, \textbf{p})$. We also have that 
\begin{equation*}
    \gamma = \frac{1}{\sqrt{1-\beta^2}},
\end{equation*}
where $\gamma$ is the Lorentz factor, and $\beta = \frac{v}{c}$, which gives us the following definitions for energy $E = \gamma m$ and momentum $\textbf{p} = \beta\gamma m$\cite{Gramstad:1631043}. 
From this we can derive the energy momentum formula:

\begin{align*} 
    \textbf{p}^2 &= \beta^2\gamma^2m^2  \\ 
    \textbf{p}^2 + m^2 &=  m^2(\beta^2\gamma^2 + 1) \\
    \textbf{p}^2 + m^2 &= m^2\gamma^2 \\
    \textbf{p}^2 + m^2 &= E^2
\end{align*}

\begin{equation}\label{eq:energy_momentum}
    E = \sqrt{p^2 + m^2}.
\end{equation}

It can be shown that the phase space of a particle is given by\cite{green_highpt}:

\begin{equation}\label{eq:phase_space}
    d\textbf{p} = dp_xdp_ydp_z = p^2dpd\Omega = dp_zp_Tdp_Td\phi,
\end{equation}

where $p_z$ is the momentum along the beam direction, $p_T$ is the projected momentum on the transverse plane, 
and $\Omega$ is the solid angle. An analog to the relativistic longitudal velocity is the rapidity y. To define this 
we have that the relativistic generalization of equation \ref{eq:phase_space} is given by:
\begin{equation*}
    d^4p\delta (E^2 - p^2 - m^2) = d\textbf{p}\frac{1}{E} = p_T dp_Td\phi dy, \, \, dy = \frac{dp_z}{E}.
\end{equation*}

Using the fact that $p = \sqrt{p_T^2 + p_z^2}$ and equation \ref{eq:energy_momentum} we can integrate $dy$ to get the rapidity:

\begin{equation*}
    \int dy = \int \frac{dp_z}{\sqrt{p_T^2 + p_z^2 + m^2}}
\end{equation*}

\begin{equation}
    y = \cosh^{-1}\left( \frac{E}{\sqrt{p_T^2 + m^2}}\right)
\end{equation}

For particles with little to no mass relative to the transverse momentum, we have that $p_T^2 + m^2 \approx p_T^2$ 
where $p_T = E\sin{(\theta)}$, which gives us the following relations:

\begin{equation*}
    \cosh(y) = \frac{1}{\sin(\theta)},
\end{equation*}
\begin{equation*}
    \sinh(y) = \frac{1}{\tan(\theta)},
\end{equation*}
\begin{equation*}
    \tanh(y) = \cos(\theta).
\end{equation*}

which can be used to show that $e^{-y} = \tan{\frac{\theta}{2}}$. From this we define the pseudorapidity $\eta$ as:
\begin{equation}
    \eta = -\ln{\left( \tan{\frac{\theta}{2}}\right)},
\end{equation}
which is, in the relativistic limit, the same as the rapidity y. A useful property of the pseudorapidity is that 
the phase space of a single particle is uniformly distributed for both $\eta$ and $\phi$, making them good features 
for controls and verification of SM MC. 
\subsection*{Data collection}

\begin{figure}[h!]
    \includegraphics[width=\linewidth]{Figures/atlas/ATLAS Detector Schematic black particles.png}
    \caption[Detector tracking of particles]{Figure describing how particles are detected at ATLAS, fetched from \href{https://cds.cern.ch/record/2770815}{	ATLAS detector slice (and particle visualisations)}, by Sascha Mehlhase \cite{Mehlhase:2770815} . }
    \label{fig:atlas_particle_detect}
\end{figure}

The features in the dataset used in this analysis are computed with or fetched from the quantities measured from the 
detector itself. Such measured properties include the momentum, energy, angles etc., 
all of which are either measured or computed based on the measurements in the 
detector. In figure \ref{fig:atlas_particle_detect} a visualization shows how
different particles move through the detector and how they are detected. For example, 
energy deposits are measured using electromagnetic and hadronic calorimeters, specifically 
designed to efficiently measure the energy of all particles interacting through the 
electromagnetic and strong interactions, respectively. Charged particles leave tracks in the 
inner tracking device. Thus, electrons leaves tracks in the inner tracker and deposit energy 
in the electromagnetic calorimeter, muons pass through the muon spectrometer 
leaving electrical signal and momentum and energy of a muon is then based on the curvature 
of the track of the muon in the muon detector. Between the calorimeters and the muon detector 
there is another magnetic field to help on the measure of the muons. Photons only deposit energy in the electromagnetic 
calorimeter. Hadrons such as protons and neutrons deposit energy in the  
hadronic calorimeters, and charged hadrons also leaves tracks in the inner tracker. This is 
shown in figure \ref{fig:atlas_particle_detect}.\par

The ATLAS detector has a few selection stages before the data is stored. 
In order to reach the highest intensity of collisions, the LHC accelerates
packets of around $10^{11}$ protons, and collides each batch every 25 nanoseconds, 
yielding a collision rate of 40 MHz\cite{Wang:2707056}. %\cite{Bernius:2707054}.



\subsubsection*{Data preparation}
During data recording at the ATLAS detector, triggers on the hardware and software level select out the 
events\footnote{An event here is defined as a collision recorded and reconstructed.} that are of 
most interest. Most of the intitial collisions are discarded with about 1 in 40000 events being 
accepted. This is because the amount of recorded events simply are too 
 high to realistically analyse and store. Also, most of the events are of no interest for BSM 
 searches anyway as they do not yield massive particles. There are therefor no large deposits 
 of energy or tracks with high transverse momentum. Instead, these collision fragments leave 
 the interaction region with very small angles between their trajectory and the beam pipe. \par
 Once the trigger selection is done, the data is reconstructed. This means that the objects in 
 the recorded events are being reconstructed into particles like jets and photons using advanced 
 software algorithms. The reconstruction is done based on the tracks and measurements in the detector, 
 but it is not perfect, and can lead to fake leptons or jets. By fake, it is meant that an object 
 might look like a lepton but is in reality a jet or vice versa\cite{Gillam:2015kta}.
 Once the reconstruction is done, further slimming of the data is done. Derivations are slimmings 
 of the data where the selection of events are further reduced to match the needs of the different 
 analysis groups. \par 
 The simulated data go through parts of the same process as the recorded data from ATLAS. These 
 events are first generated by software that simulates the initial interaction and then run through 
 the detector to simulate the interaction of the particles with the detector and the resulting 
 production of digital signals. Since these signals mimic the real detector signals, the events can 
 be reconstructed and go through derivations just like the proton-proton collision data, 
and then be used in analysis. 

\begin{figure}[h!]
    \includegraphics[width=\linewidth]{Figures/atlas/data_col_phys.png}
    \caption[Steps from data collection to physics results]{Figure describing the steps to take for data collection at ATLAS, fetched from \href{https://indico.cern.ch/event/1159574/timetable/?view=standard}{Hybrid ATLAS Induction Day + Software Tutorial workshop}, part
    \href{https://indico.cern.ch/event/860971/contributions/3672974/attachments/1972049/3280896/Atlas_computing_data_preparation_jan20.pdf}{Computing and Data preparation}, 
    initially drawn by Dr. James Catmore, later held by S.M Wang \cite{Wang:2707056}. }
    \label{fig:atlas_data_col_phys}
\end{figure}


\subsubsection*{Jets}
Photons and electrons are detected in the electromagnetic calorimeter, and are easy to track and detect as they 
separate easily. Quarks, however, are bound by QCD and thus cannot be seperated as individual 
particles. An illustration of how quarks and gluons materialise as jets during a proton-proton 
collision is shown below in figure \ref{fig:cms_jets}.

\begin{figure}[h!]
    \includegraphics[width=\linewidth]{Figures/atlas/cms_Sketch_PartonParticleCaloJet.png}
    \caption[Jet produciton from pp-collisions to detector]{Figure describing how quarks and gluons are treated in the detector, and thus why we name them jets, fetched from \href{https://cms.cern/sites/default/files/field/image/Sketch_PartonParticleCaloJet.png}{the CMS webpage}. }
    \label{fig:cms_jets}
\end{figure}

In a proton-proton collision, the quarks and gluons form stable or unstable hadrons such 
that the color confinement is upheld\cite{Hwang_Wu_2018}. These particles then decay to other stable 
hadrons that can be tracked, and these collimated tracks and energy deposits are called jets. The jets are defined through
complex algorithms. Another point to make is that some quarks yield more 
information than others. Using the properties of heavy quarks such as the b- and c-quarks and 
their non-negligible lifetime allows us to differentiate jets containing these quarks from 
jets only consisting of light quarks. The b-jet coming from a b quark are particularly interesting 
as they are produced in the decay of top quarks. The Higgs can be produced from, 
and in association with, top-quarks. The behavior of hadrons containing b-quarks may be 
indirectly sensitive to BSM physics as well. 