\section{Final remarks on the results}
In sections \ref{sec:nonsig}, \ref{sec:3lep}, \ref{sec:2lep} the performance 
of the regular and variational autoencoder with respect to usage in BSM 
searches has been shown through various tests. The final remarks can be summed 
up in four points:
\begin{itemize}
    
    \item Shape of SM MC reconstruction error distribution 
    \item Dataset altering testing
    \item Megaset changes 
    \item Network architecture

\end{itemize}
First, it appears that the shape of the reconstruction error is more sensitive 
to the choice between variational and regular autoencoder than it is to the 
amount of training samples it uses. However, the steepness of the shape in the 
case of the regular autoencoder increases with the increase of training data. 
From this obervation it is reasonable to assume that to increase the ability of 
the autoencoder to perform, one needs large amounts of training samples. This 
is also shown in the case of the variational autoencoder, where if one increases 
the amount of training data by going from the 3 lepton + $e_T^{miss}$ case to 
the 2 lepton + $e_T^{miss}$ case we observe the peak to move more to the left 
end of the reconstruction error distribution. It was not well investigated 
why the outputs of the two models are so different, but one reason could be 
that the decoder that sampels from the latent space distribution in the 
variational autoencoder needs even more training data to get good separation. \par 
Secondly, with the increase in training data it would be interesting to do the 
"altering $p_T$" test, as well as some other SM MC altering tests with especially the 
regular autoencoder. Based on the signal tests and the "altering $p_T$" test done with 
the 3 lepton + $e_T^{miss}$ trained regular autoencoder, it is reasonable that the 
autoencoder would perform even better. Other tests could for example be to swap 
certain features in the RMM in order to create unphysical events, or swap events from
one decay channel with another in order to create unphysical events. Essentially, 
by making these anomalous events and testing the autoencoder, one would get a better 
picture on the reach and ability of the autoencoder. This is of importance given the 
fact that the target signal or signals could in theory look very strange or perhaps 
in some feature space very similar to the SM MC. \par 
A third point to note is in regard to the megaset training done in the 2 lepton + 
$e_T^{miss}$ case. An arbitrary choice of 10 megasets were chosen, but it might 
be better to choose a larger or smaller number of megasets, and that might have an impact
on how well the autoencoders learn the SM signatures. A key criterion to uphold when 
doing that training is to ensure that the megasets all keep the natural distribution of 
the entire training set, and thus the standard model. If not, the algorithm will learn
with a bias that is not contructive. \par 
The fourth point to draw from the results are that it appears that the results are not 
too sensitive to the architecture of the networks. The two different architectures chosen
were one model with only a latent space layer between the input and output layer, and a model
with three layers on each side of the latent space. The choice of layers and nodes per layer 
was somewhat arbitrary, and after inference it was also clear that really only the latent space
was sensitive to change with respect to performance in reconstruction error. This is though 
more related to the size of the input layer and the complexity of the data, more than 
the number of events in the training and test set. 