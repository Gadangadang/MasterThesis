\section{Challenges with the search method and tools}

In the previous sections, the output and results of using the autoencoder for anomaly detection have been shown. 
The method and results, as produced and shown in this thesis, have yielded some promising results, given the nature 
of the search method. However, it should be known what the challenges of the task actually are, to truly understand 
why the results only are promising, and not great compared to other search methods. The challenges can be divided into 
three main points, all of which are entangled together. The three challenges are listed below. 

\begin{itemize}
    \item Model independence
    \item Reconstruction error minimization
    \item Feature engineering
\end{itemize}

Model independence is the first challenge. By model it is here referring to signal models, which it self might be self 
explanatory, but is a incredibly difficult criteria to uphold. As mentioned in the theory section for the Standard Model, 
the Standard Model, all though very successful in certain predictions, lack the ability to explain a whole number of 
behavior around us, and so there have been made a large numbers of suggested solutions to the issues. These new models are 
often called extensions to the Standard Model, and all though mathematically consistent, not neccesarly physically possible. 
And even if they are physically possible, in other words, they adhere to certain fundamental physical principles, they 
still might not exsist, as several searches at ATLAS have exluded but never found any new physics. The search method is inherently 
biased as one assumes that the new physics looks like the signal, and thus do analysis, data preparation etc with that 
signal in mind. But we do not know, at all, what the new physics looks like, even the assumption that we are looking for 
particles are implying a bias that might not be true. From the collisions in the detector to the analysis, there are biased descisions, 
we cannot avoid them, but we can minimize them as much as possible, which is a goal with the search strategy in this thesis. \par
This leads us to the second challenge, which is reconstruction error minimization. The proposed method in this thesis is to 
learn the signature of the standard model so well, that even subtle anomalous behavior will be picked up by the analysis tool. 
The autoencoder learns the signature of the standard model, by mimimizing the reconstruction error, and then hopefully, the 
anomalous data will be picked up and skewed to the right end of the reconstructino error distribution in a signal region. 
One problem with this is that one first blinds oneself to signals that might be very, very similar to the standard model in some 
feature space, but with very low statistics. These events will for a given set of features, never be found. \par 
The The third challenge is the choice of features. This thesis utilized the RMM structure by Chekanov et al, as it maximizes 
the amount of information in the input data by using almost completely uncorrolated features. However, as we do not know the signals 
we are looking for, there is no way to know if this choice is the optimal choice for new physics. In fact, even if we found 
an ideal set of features, based on some physical principles or something else, it is not trivial that the reconstruction error calculation
should weight the error of each feature equally. It might very well be that some features are less important than others. Essentially, 
the goal is to optimize for a signal we do not know, using features we dont know are optimal, and weighting them as "unbiased" as 
possible, simply taking the average, to dictate how the autoencoder learns and updates its internal weights and biases. 
To be blunt, this exercise is the equivalent of fishing in the dark, with a blindfold on, with a net of an arbitrary size, 
in an ocean trying to not catch fish, but something else that may or may not look like fish that might or might not be in the ocean. 
