\chapter*{Abstract} 
This master thesis investigates the performance and usage of autoencoders in Beyond Standard Model (BSM) searches, 
using n-tuples from ATLAS that are converted to python dataframe structures. The Rapidity-Mass matrix (RMM) is proposed 
as the input data features, with 6 b-jets and 6 l-jets, and 5 of each lepton. The goal was to test and understand 
the performance in the 3 lepton + $e_T^{miss}$ final state, but due to poor results, the 2 lepton + $e_T^{miss}$ 
dataset was used instead. To deal with the large dataset size, megasets were used to conserve the overall distribution 
in smaller batches. The autoencoders were benchmarked by creating anomalous events by altering the $p_T$ of standard 
model events and testing on two supersymmetric signal models. The performance was measured in three categories: 
reconstruction error, background and signal reduction, and significance when performing cuts. The regular 
autoencoder performed better in the first and second categories, while the variational autoencoder performed slightly 
better in the third. Increasing the training data improved the performance in all categories. A blind test 
with ATLAS data and BSM signals showed that the autoencoder, when trained on the 2 lepton + $e_T^{miss}$ dataset, 
managed to separate out some signals from the ATLAS data, indicating its potential use in more rigorous analyses. 
The thesis also discusses future work and challenges, such as computational bottlenecks, further investigation of 
the RMM, and better feature engineering.