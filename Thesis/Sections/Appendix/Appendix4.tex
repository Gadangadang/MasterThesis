\section{The Mellin Transform}\label{sec:Appendix Mellin Transform}
For a function $f(x)$ defined on the positive real axis, the Mellin transformation $\mathcal{M}$ is the operation mapping $f$ into the function $\Me{f}$ defined on the complex plane. It has the following definition
\begin{align}
    \mathcal{M}\big[f(x):N\big]=\tilde{f}(N)=\int_{0}^{\infty}dx\,x^{N-1}\,f(x)\,,
\end{align}
where $\Me{f}(N)$ is the Mellin transform of $f(x)$, and $N$ is the Mellin moment conjugate to $x$. In general, the integral does not exist for all functions, i.e. all functions does not have a well defined Mellin transform. But even if the transform exist, it is not guaranteed to converge. The domain of $N$ for which the integral converge for a given function is known as the fundamental strip. For a real function, the fundamental strip is denoted as $<a,b>$, given by all points on the domain $a<s<b$ such that $N=s+it$, for any t. The values of $a$ and $b$ is found by the asymptotic behaviour
\begin{align}
    a:\hspace{0.5cm}\lim_{x\rightarrow 0^{+}}f(x)&=\mathcal{O}(x^{-a})\,,
    \\
    b:\hspace{0.5cm}\lim_{x\rightarrow\infty}f(x)&=\mathcal{O}(x^{-b})\,,
\end{align}
which implies that a function defined on the domain $0<x<1$, have a fundamental strip $<a,\infty>$.

The Mellin transform is closely related to the two-sided Laplace transform, with the difference of a variable change $x\rightarrow -\ln x$. Hence, the inverse Mellin is given by
\begin{align}\label{eq:Appendix Inverse Mellin}
    \mathcal{M}^{-1}\big[\Me{f}(N):x\big]=\frac{1}{2\pi i}\int_{c-i\infty}^{c+i\infty}dN\,x^{-N}\Me{f}(N)\,,
\end{align}
where the integration contour is along a vertical line through $Re(N)=c$, as long as $c$ lies in the fundamental strip of the function. If the function is holomorphic in the strip and vanishes sufficiently fast when $Im(N)\rightarrow\pm\infty$, it follows from Cauchy's theorem that the contour may be deformed as long as no poles are crossed. 

One very important property of the Mellin transform is its effect on convolutions
\begin{align}\label{eq:Appendix Mellin convolution}
    \big(f\star g\big)(x)=\int_{0}^{1}dx_1\int_{0}^{1}dx_2\,f(x_1)g(x_2)\delta(x-x_1x_2)=\int_{x}^{1}\frac{dx_1}{x_1}\,f(x_1)g\big(\frac{x_2}{x_1}\big)\,,
\end{align}
where $x\in(0,1)$. To disentangle this convolution, one performs the transform
\begin{align}\label{eq:Appendix Mellin convolution transform}
    \mathcal{M}\big[\big(f\star g\big)(x):N\big]&=\int_{0}^{1}dx\,x^{N-1}\int_{0}^{1}dx_1\int_{0}^{1}dx_2\,f(x_1)g(x_2)\delta(x-x_1x_2)\nonumber
    \\
    &=\int_{0}^{1}dx_1\,x_1^{N-1}f(x_1)\int_{0}^{1}dx_2\,x_2^{N-1}g(x_2)\nonumber
    \\
    &=\Me{f}(N)\Me{g}(N)\,,
\end{align}
where the convolution in $x$-space has transformed into simple products in Mellin space. 

\subsection{Mellin Transforms of Functions}
This section is intended to demonstrate some of the Mellin transforms that are encountered in this thesis. Since the main focus is in the domain of large $N$, this will not be the general treatment of these transforms.

The simplest case is the Mellin transform of a constant $c$,
\begin{align}
    \int_{0}^{1}dx\,x^{N-1}c=\frac{c}{N}\,,
\end{align}
and a monomial
\begin{align}
    \int_{0}^{1}dx\,x^{N-1}x^{a}=\frac{1}{N+a}\,,
\end{align}
which combined with a logarithm
\begin{align}
    \int_{0}^{1}dx\,x^{N-1}(-x^{a}\ln x)&=-\frac{x^{N+a}}{N+a}\ln x\big|_{0}^{1}+\int_{0}^{1}dx\,\frac{x^{N+a-1}}{N+a}\nonumber
    \\
    &=\frac{1}{(N+a)^{2}}\,,
\end{align}
which can be generalized for any polynomial $P(x)$
\begin{align}
    \int_{0}^{1}dx\,x^{N-1}P(x)=\mathcal{O}(1/N)\,.
\end{align}
Furthermore, by using that $x^{N-1}=e^{(N-1)\ln x}$, repeated derivatives with respect to $N$ will give
\begin{align}
    \int_{0}^{1}dx\,x^{N-1} f(x)\ln^{k}x=\frac{d^{k}}{dN^{k}}\Me{f}(N),\hspace{1cm}\forall\,k>0\,.
\end{align}
Another useful property involves the derivative of a function
\begin{align}\label{eq:Appendix derivative of function}
    \int_{0}^{1}dx\,x^{N-1}\Big(-x\frac{d}{dx}f(x)\Big)&=-x^{N}f(x)\big|_{0}^{1}+N\int_{0}^{1}dx\,x^{N-1}f(x)\nonumber
    \\
    &=-f(1)+N\Me{f}(N)\,,
\end{align}
which is especially useful for functions that vanish for $x=1$.

The Mellin transform of plus distributions are more complicated. By using \cref{eq:plus distribution convolution}, we can write
\begin{align}
    \int_{0}^{1}dx\,x^{N-1}\Big[\frac{1}{1-x}\Big]_{+}=\int_{0}^{1}dx\Big(\frac{x^{N-1}}{1-x}-\frac{1}{1-x}\Big)\,.
\end{align}
The terms on the $rhs$ diverge when considered separately, so we can not calculate them independently. By introducing a regulator $\epsilon$, this can be rewritten by using beta integrals
\begin{align}
    \int_{0}^{1}dx \frac{x^{N-1}}{(1-x)^{1-\epsilon}}-\int_{0}^{1}dx\frac{1}{(1-x)^{1-\epsilon}}=\frac{\Gamma(N)\Gamma(\epsilon)}{\Gamma(N+\epsilon)}-\frac{\Gamma(1)\Gamma(\epsilon)}{\Gamma(1+\epsilon)}\,,
\end{align}
which by the recursion relation $x\Gamma(x)=\Gamma(x+1)$, can be shown to give\footnote{After the regulator has been removed.} 
\begin{align}
    \int_{0}^{1}dx\,x^{N-1}\Big[\frac{1}{1-x}\Big]_{+}&=\sum_{k=1}^{N-1}\frac{1}{k}\nonumber
    \\
    &=-\Big(\int_{1}^{N}dk\frac{1}{k}+\lim_{N\to\infty}\Big(\sum_{k=1}^{N-1}\frac{1}{k}-\int_{1}^{N}dk\frac{1}{k}\Big)+\mathcal{O}(1/N)\Big)\nonumber
    \\
    &=-\ln\Bar{N}+\mathcal{O}(1/N)\,,
\end{align}
where $\Bar{N}=Ne^{e^{\gamma_{E}}}$. 

Particularly useful moments are those of plus distributions with logarithms, see \cite{CATANI1989} 
\begin{align}
    \int_{0}^{1}dx\,x^{N-1}\Big[\frac{\ln(1-x)}{1-x}\Big]_{+}&=\frac{1}{2}\ln^{2}\Bar{N}+\frac{1}{2}\zeta(2)+\mathcal{O}(1/N)\,,\label{eq:App mellin of ln plus dist}
    \\
    \int_{0}^{1}dx\,x^{N-1}\Big[\frac{\ln^{2}(1-x)}{1-x}\Big]_{+}&=-\frac{1}{3}\ln^{3}\Bar{N}-\zeta{2}\ln\Bar{N}-\frac{2}{3}\zeta{3}+\mathcal{O}(1/N)\,,
\end{align}
where $\zeta(y)$ is the Riemann zeta function. In the large $N$ limit, the following behaviour follows
\begin{align}
    \int_{0}^{1}dx x^{N-1}\Big[\frac{\ln^{n}(1-x)}{(1-x)}\Big]_{+}=\frac{(-1)^{n+1}}{n+1}\ln^{n+1}(\bar{N})+\mathcal{O}(\ln^{n-1}(\bar{N}))\,.
\end{align}